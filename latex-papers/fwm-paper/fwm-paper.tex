\documentclass[12pt,a4paper]{article}

% Packages for math, symbols, and formatting
\usepackage{amsmath, amssymb, amsfonts}  % Standard math packages
\usepackage{graphicx}                     % For figures
\usepackage{hyperref}                     % For links
\usepackage{geometry}                     % Page layout
\usepackage{caption}                      % Captions for figures/tables
\usepackage{bm}                           % Bold math symbols

% Page layout
\geometry{top=2.5cm, bottom=2.5cm, left=2.5cm, right=2.5cm}

% Numbering equations by section
\numberwithin{equation}{section}

% Title and authors
\title{The general analytic solution to continuous wave four-wave mixing in nonlinear fiber optics in Weierstrass notation}
\author{Graham Hesketh \\
\small United Kingdom \\
\small \texttt{gdh1e10@gmail.com}}
\date{\today}

\begin{document}

\maketitle

\begin{abstract}
The general analytic solution to continuous wave four-wave mixing in nonlinear optical fibers is presented in terms of Weierstrass elliptic $\wp, \sigma, \zeta$ functions. 
Solutions are provided for the full complex envelopes for all four frequency modes, under all initial conditions, and without any undepleted pump approximation.
\end{abstract}


% ---------- SECTION -------------
% --------------------------------

\section{Introduction}
Four-wave mixing (FWM) in nonlinear optical fibers is one of the fundamental parametric processes enabled by the Kerr nonlinearity. 
It underpins a wide range of applications, including wavelength conversion, parametric amplification, frequency-comb generation, and quantum light sources. 
Despite this broad relevance, obtaining closed-form analytic descriptions of FWM remains challenging because the underlying coupled-wave equations are nonlinear, phase-sensitive, and generally require numerical integration. 
As a result, most textbook treatments rely on simplifying assumptions such as undepleted pumps, negligible phase mismatch, or weak signal and idler powers.

Several works have presented analytic or semi-analytic solutions in specific regimes, including the undepleted-pump limit, the perfectly phase-matched case, or configurations with constrained input conditions. 
However, a fully general analytic solution—valid for arbitrary pump depletion, arbitrary phase mismatch, and arbitrary input power ratios—remains of significant theoretical and practical interest. 
Such a solution not only clarifies the structure of the FWM interaction but also provides a unified benchmark against which approximate models and numerical simulations can be evaluated.

In this work, we derive the general analytic solution for continuous-wave four-wave mixing in a nonlinear optical fiber. 
Beginning from the standard coupled-wave equations for the interacting fields, we identify the conserved quantities associated with the parametric interaction and use them to reduce the system to an integrable form. 
The resulting expressions describe the full evolution of the pump, signal, and idler amplitudes, including amplitude and phase dynamics, for arbitrary initial conditions. 
These solutions recover the known limiting cases and offer direct physical insight into gain behavior, conversion efficiency, and phase evolution across the full parameter space.


% ---------- SECTION -------------
% --------------------------------

\section{The continuous wave four-wave mixing system}
The following coupled system of ordinary differential equations is taken from Agrawal and describes four-wave mixing in the continuous wave limit, i.e., in the absence of time derivatives:

\begin{align}
\frac{dA_1}{dz} &= \frac{in_2\omega_1}{c}\left[\left(f_{11}|A_1|^2 + 2\sum_{k\neq 1} f_{1k}|A_k|^2\right)A_1 + 2f_{1234}A_2^*A_3A_4e^{i\Delta kz}\right], \notag \\
\frac{dA_2}{dz} &= \frac{in_2\omega_2}{c}\left[\left(f_{22}|A_2|^2 + 2\sum_{k\neq 2} f_{2k}|A_k|^2\right)A_2 + 2f_{2134}A_1^*A_3A_4e^{i\Delta kz}\right], \notag \\
\frac{dA_3}{dz} &= \frac{in_2\omega_3}{c}\left[\left(f_{33}|A_3|^2 + 2\sum_{k\neq 3} f_{3k}|A_k|^2\right)A_3 + 2f_{3412}A_1A_2A_4^*e^{-i\Delta kz}\right], \notag \\
\frac{dA_4}{dz} &= \frac{in_2\omega_4}{c}\left[\left(f_{44}|A_4|^2 + 2\sum_{k\neq 4} f_{4k}|A_k|^2\right)A_4 + 2f_{4312}A_1A_2A_3^*e^{-i\Delta kz}\right]. \label{eq:fwm-system}
\end{align}

where:
\begin{itemize}
    \item $A_j$ are the slowly-varying complex field amplitudes for waves $j = 1, 2, 3, 4$
    \item $z$ is the propagation distance
    \item $\omega_j$ are the angular frequencies of the respective waves
    \item $c$ is the speed of light in vacuum
    \item $n_2$ is the nonlinear refractive index
    \item $f_{jj}$ are the self-phase modulation (SPM) coefficients
    \item $f_{jk}$ (for $j \neq k$) are the cross-phase modulation (XPM) coefficients
    \item $f_{jklm}$ are the four-wave mixing (FWM) coefficients
    \item $A^*$ denotes complex conjugation
    \item $\Delta k = \beta_1 + \beta_2 - \beta_3 - \beta_4$ is the phase mismatch
    \item $\beta_j = n(\omega_j)\omega_j/c$ are the propagation constants
\end{itemize}

The modal overlap integrals are defined in terms of the transverse distribution of the fiber mode $F_j(x,y)$ as:
\begin{align}
\label{eq:overlaps}
f_{jklm} &= \frac{\langle F_j^* F_k^* F_l F_m \rangle}{\sqrt{\langle |F_j|^2 \rangle \langle |F_k|^2 \rangle \langle |F_l|^2 \rangle \langle |F_m|^2 \rangle}}, \\
f_{jk} &= f_{kj} = f_{jjkk}
\end{align}
where $\langle \cdots \rangle = \iint_{-\infty}^{\infty} (\cdots) \, dx\,dy$ denotes the transverse spatial integral.

Agrawal says of \eqref{eq:fwm-system} that the equations ``are quite general in the sense that they include the effects of SPM, XPM, and pump depletion on the FWM process; a numerical approach is necessary to solve them exactly.''
That said, herein, they are solved analytically in full, as written, without any further approximation.


% ---------- SECTION -------------
% --------------------------------

\section{Simplifying parameter dependence}
From \eqref{eq:overlaps}, we observe the following symmetries among the wave mixing coefficients:

\begin{align}
f_{1234} &= |f_{1234}|e^{i\theta}, \notag \\
f_{2134} &= f_{1234} = |f_{1234}|e^{i\theta}, \notag \\
f_{3412} &= f_{1234}^* = |f_{1234}|e^{-i\theta}, \notag \\
f_{4312} &= f_{1234}^* = |f_{1234}|e^{-i\theta}, \label{eq:overlap-symms}
\end{align}
where $\theta$ is the phase of $f_{1234}$. 
Consequently, we can conveniently renormalise the functions, and also absorb the phase $\theta$ as a global phase rotation of the modes, 
in such a way that the wave mixing coefficients all become equal to one.
To do so, we introduce the following redefinition of the mode functions 
(take care to note that, while not crucial, the complex conjugates are shared among $u,v$ in this choice of labels so as to later conveniently give one product over $u$ and one over $v$):

\begin{align}
T &= \sqrt{\frac{2\,n_2 \left|f_{1234}\right|}{c} \sqrt{ \prod\limits_{k=1}^{4} \omega_k}}, \label{eq:T-def} \\
u_1\left(z\right) &= \frac{T\,\,e^{-i\pi/4}\,e^{-i\theta/4}}{\sqrt{\omega_{1}}} \,A_{1}\left(z\right)\,e^{i z \beta_{1}}, \notag \\
u_2\left(z\right) &= \frac{T\,e^{-i\pi/4}\,e^{-i\theta/4}}{\sqrt{\omega_{2}}}\,A_{2}\left(z\right)\,e^{i z \beta_{2}}, \notag \\
u_3\left(z\right) &= \frac{T\,e^{i\pi/4}\,e^{-i\theta/4}}{\sqrt{\omega_{3}}}\,A^*_{3}\left(z\right)\,e^{-i z \beta_{3}}, \notag \\
u_4\left(z\right) &= \frac{T\,e^{i\pi/4}\,e^{-i\theta/4}}{\sqrt{\omega_{4}}}\,A^*_{4}\left(z\right)\,e^{-i z \beta_{4}}, \notag \\
v_1\left(z\right) &= \frac{T\,e^{-i\pi/4}\,e^{i\theta/4}}{\sqrt{\omega_{1}}}\,A^*_{1}\left(z\right)\,e^{-i z \beta_{1}}, \notag \\
v_2\left(z\right) &= \frac{T\,e^{-i\pi/4}\,e^{i\theta/4}}{\sqrt{\omega_{2}}}\,A^*_{2}\left(z\right)\,e^{-i z \beta_{2}}, \notag \\
v_3\left(z\right) &= \frac{T\,e^{i\pi/4}\,e^{i\theta/4}}{\sqrt{\omega_{3}}}\,A_{3}\left(z\right)\,e^{i z \beta_{3}}, \notag \\
v_4\left(z\right) &= \frac{T\,e^{i\pi/4}\,e^{i\theta/4}}{\sqrt{\omega_{4}}}\,A_{4}\left(z\right)\,e^{i z \beta_{4}}. \label{eq:u-v-A}
\end{align}

We subsequently redefine the remaining phase modulation parameters as:
\begin{align}
{a}_{j} &= - i\, s{\left(j \right)} {\beta}_{j}, \\
{a}_{j,k} &= {a}_{k,j} = - \frac{\left(\delta_{j k} - 2\right) s{\left(j \right)}\, s{\left(k \right)}\, {f}_{j,k}\, {\omega}_{j}\, {\omega}_{k}}{2 \left|{{f}_{1,2,3,4}}\right| \sqrt{\prod_{l=1}^{4} {\omega}_{l}}}
\end{align}
with Kronecker $\delta$, and $s(j)$ defined such that $s(1) = s(2) = 1, s(3) = s(4) = -1$.

The four-wave mixing system in \eqref{eq:fwm-system} is thus recast as:
\begin{align}
\frac{d}{d z} u_j{\left(z \right)} &= - \left({a}_{j} + \sum_{k=1}^{4} {a}_{j,k}\,u_k v_k \right) u_j + \prod\limits_{k=1, k \ne j}^{4} v_k, \notag \\
\frac{d}{d z} v_j{\left(z \right)} &= \left({a}_{j} + \sum_{k=1}^{4} {a}_{j,k}\,u_k v_k \right) v_j - \prod\limits_{k=1, k \ne j}^{4} u_k \label{eq:uv-system}
\end{align}


% ---------- SECTION -------------
% --------------------------------

\section{Conserved quantities}

Based on prior experience of similar systems we are prompted to ask if the system in \eqref{eq:uv-system} is a canonical Hamiltonian system, and indeed we confirm that it is with:
\begin{align}
\label{eq:ham-uv}
H(u_1,\dots, u_4, v_1, \dots, v_4) &= -\sum_{j=1}^{4} a_{j}\, u_j v_j - \frac{1}{2}\,\sum_{j,k=1}^{4} a_{j,k}\, u_j v_j u_k v_k + \prod\limits_{j=1}^{4} u_j + \prod\limits_{j=1}^{4} v_j, \\
\frac{d}{d z} u_j{\left(z \right)} &= \frac{\partial H}{\partial v_j}, \\
\frac{d}{d z} v_j{\left(z \right)} &= -\frac{\partial H}{\partial u_j}, \\
\frac{d}{d z} H &= 0
\end{align}

The conservation of $H$ is to be expected as a consequence of \eqref{eq:ham-uv} lacking an explicit $z$ dependence. The pair $u_j, v_j$ are Hamiltonian conjugates, and each pair represents one of four degrees of freedom in the system. 
We refer to their product $u_j v_j$ as the modal power and it evolves according to:
\begin{equation}
\label{eq:duv_j} 
\frac{d}{d z} u_j v_j = \prod\limits_{k=1}^{4} v_k - \prod\limits_{k=1}^{4} u_k
\end{equation}

As the right hand side of \eqref{eq:duv_j} is the same for all $j$, 
we may define constants $\gamma_j$ and function $\rho(z)$ such that:
\begin{equation}
\label{eq:uvj-gam-rho}
u_j(z) v_j(z) = \gamma_j - \rho(z)
\end{equation}

from which it follows that there are 3 (read as $j > k$ to avoid over counting) intermodal power conservation laws of the form:
\begin{equation}
\label{eq:power-conserved}
u_j(z) v_j(z) - u_k(z) v_k(z) = \gamma_j - \gamma_k.
\end{equation}

The system has 4 degrees of freedom and four conserved quantities which is a requirement for integrability in the Liouville-Arnold sense.
The $\gamma_j$ constants can be determined from \eqref{eq:power-conserved} and initial conditions, 
however, this provides three equations for four unknowns and thus a choice is available for normalisation. 
We choose to impose the constraint that:
\begin{equation}
\sum\limits_{j=1}^{4}\gamma_j = 0.
\end{equation} 

Remark: Appendix-\ref{app:fs-det-ham-proof} shows that \eqref{eq:ham-uv} is the Frobenius-Stickelberger determinant.


% ---------- SECTION -------------
% --------------------------------

\section{Solutions for modal powers in terms of Weierstrass \texorpdfstring{$\wp$}{wp} elliptic functions}

One key thing to note in order to obtain elliptic functions solutions is that the derivative of the modal power in \eqref{eq:duv_j} is proportional to the difference of the two wave mixing product terms, and that the Hamiltonian in \eqref{eq:ham-uv} contains their sum.
We then note the simple but important identity:
\begin{equation}
\label{eq:prod-id}
\left(\prod\limits_{k=1}^{4} v_k - \prod\limits_{k=1}^{4} u_k\right)^2 - \left(\prod\limits_{k=1}^{4} v_k + \prod\limits_{k=1}^{4} u_k\right)^2 = - 4 \prod\limits_{j=1}^{4} u_j v_j
\end{equation}
which enables us to square \eqref{eq:duv_j} and replace wave mixing terms with phase modulation terms.

To proceed in this manner let us introduce the function $Q$ which represents the phase modulation part of the Hamiltonian:
\begin{align}
Q(u_1(z)v_1(z), \dots, u_4(z)v_4(z)) &= a_0 + \sum_{j=1}^{4} a_{j}\, u_j v_j + \frac{1}{2}\,\sum_{j,k=1}^{4} a_{j,k}\, u_j v_j u_k v_k , \notag  \\
& = \sum_{l=0}^{2}b_l\,\rho\left(z\right)^l \label{eq:Q-uv}
\end{align}

with $a_0=H$, and observe that by squaring \eqref{eq:duv_j} and substituting \eqref{eq:ham-uv}, \eqref{eq:uvj-gam-rho}, \eqref{eq:prod-id}, and \eqref{eq:Q-uv}, we obtain:
\begin{align}
\left(\frac{d}{d z} \rho{\left(z \right)}\right)^{2} &= Q^{2}{\left({\gamma}_{1} - \rho{\left(z \right)}, \dots, {\gamma}_{4}  - \rho{\left(z \right)}\right)} - 4 \prod_{j=1}^{4} \left({\gamma}_{j} - \rho{\left(z \right)}\right), \label{eq:drho-sqrd-1} \\
\left(\frac{d}{d z} \rho{\left(z \right)}\right)^{2} &=  \left(\sum_{l=0}^{2}b_l\,\rho\left(z\right)^l\right)^{2} - 4 \prod_{j=1}^{4} \left({\gamma}_{j} - \rho{\left(z \right)}\right), \label{eq:drho-sqrd-2} \\
\left(\frac{d}{d z} \rho{\left(z \right)}\right)^{2} &= \sum_{l=0}^{4}d_l\,\rho\left(z\right)^l, \label{eq:drho-sqrd-3} \\
\left(\frac{d}{d z} \rho{\left(z \right)}\right)^{2} &= d_4 \prod_{l=1}^{4} \left(\rho\left(z\right) -  \lambda_l\right), \label{eq:drho-sqrd-4}
\end{align}

where $b_l$, and $d_l$ are given in terms of other parameters and initial conditions in Appendices \ref{app:param-def-appendix} and \ref{app:init-conds}, and where $\lambda_l$ are the roots of the quartic polynomial in $\rho(z)$, i.e., $\sum_{l=0}^{4}d_l\,\lambda^l=0$.
We now transform \eqref{eq:drho-sqrd-4} from quartic to the standard cubic form of the $\wp$ function using the classical trick which can be conceptualised in three steps:
\begin{align}
\rho\left(z\right) &= q\left(z\right) + \lambda_1 & \text{shift so rhs is 0 for $q(z)=0$}, \notag \\
s\left(z\right) &= 1/q\left(z\right) & \text{invert to make the quartic a cubic}, \notag \\
w\left(z\right) &= C_1\,s\left(z\right) + C_0 & \text{shift and scale with $C_j$ to match Weierstrass coefficients}. \label{eq:wp-4-to-3}
\end{align}

The procedure sketched in \eqref{eq:wp-4-to-3} is implemented in the following single transformation:
\begin{align}
\rho{\left(z \right)} &= {\lambda}_{1} + \frac{d_4}{- 4 w{\left(z \right)} \prod_{l=1}^{3} {\Omega}_{l}  + \frac{d_4}{3} \sum_{l=1}^{3} {\Omega}_{l} }, \label{eq:rho-wp-1} \\
\Omega_l &= \frac{1}{\lambda_{l+1} - \lambda_1}, \notag \\
\left(\frac{d}{d z} w{\left(z \right)}\right)^{2} &= 4\,w{\left(z \right)}^3 - g_2\,w{\left(z \right)} - g_3, \label{eq:dwp} \\
g_{2} &= {d}_{0} {d}_{4} - \frac{{d}_{1} {d}_{3}}{4} + \frac{{d}_{2}^{2}}{12}, \label{eq:g2-d} \\
g_{3} &= \frac{{d}_{0} {d}_{2} {d}_{4}}{6} - \frac{{d}_{0} {d}_{3}^{2}}{16} - \frac{{d}_{1}^{2} {d}_{4}}{16} + \frac{{d}_{1} {d}_{2} {d}_{3}}{48} - \frac{{d}_{2}^{3}}{216} \label{eq:3-d}
\end{align}
where the constants $g_2$ and $g_3$ are known as Weierstrass elliptic invariants. 

Equation \eqref{eq:dwp} defines the Weierstrass elliptic $\wp$ function and the solution is:
\begin{equation}
w(z) = \wp (z - z_0, g_2, g_3)
\end{equation}
where $z_0$ is a constant chosen to match initial conditions. This constant $z_0$, and others that we will now introduce, can be obtained by inverting $\wp$ using an elliptic integral, (e.g. Carlson's symmetric $R_F$ integral). 
As $\wp$ is an even function, it is neccessary to also specify a corresponding condition for the derivative when inverting, i.e., to find $z$ from known $x,y$ we give conditions such as $\wp\left(z\right)=x, \wp'\left(z\right)=y$, where $\wp'$ is the derivative of $\wp$ known as Weierstrass p-prime. 
The points obtained during such an inversion are determined modulo the period lattice of the doubly periodic $\wp$. 
Let us proceed to define the point $z_0$, and introduce the point $z_1$ as the pole of $\rho(z)$, and points $\mu_j$ as zeroes of $u_j(z)v_j(z)$:
\begin{align}
\wp{\left(z_{0}\right)} &= \frac{{d}_{2}}{12} + \frac{{d}_{3} {\lambda}_{1}}{4} + \frac{{d}_{4} {\lambda}_{1}^{2}}{2} + \frac{- {d}_{1} - 2 {d}_{2} {\lambda}_{1} - 3 {d}_{3} {\lambda}_{1}^{2} - 4 {d}_{4} {\lambda}_{1}^{3}}{4 \left(- \rho{\left(0 \right)} + {\lambda}_{1}\right)}, \notag \\
\wp'{\left(z_{0}\right)} &= \frac{\left({d}_{1} + 2 {d}_{2} {\lambda}_{1} + 3 {d}_{3} {\lambda}_{1}^{2} + 4 {d}_{4} {\lambda}_{1}^{3}\right) }{4 \left(\rho{\left(0 \right)} - {\lambda}_{1}\right)^{2}}\left. \frac{d}{d z} \rho{\left(z \right)} \right|_{\substack{ z=0 }}, \notag \\
\wp{\left(z_{1}\right)} &= \frac{{d}_{2}}{12} + \frac{{d}_{3} {\lambda}_{1}}{4} + \frac{{d}_{4} {\lambda}_{1}^{2}}{2}, \notag \\
\wp'{\left(z_{1}\right)} &= \frac{\left(- {d}_{1} - 2 {d}_{2} {\lambda}_{1} - 3 {d}_{3} {\lambda}_{1}^{2} - 4 {d}_{4} {\lambda}_{1}^{3}\right) \sqrt{{d}_{4}}}{4}, \notag \\
\wp{\left({\mu}_{j} - z_{0}\right)} &= \frac{{d}_{2}}{12} + \frac{{d}_{3} {\lambda}_{1}}{4} + \frac{{d}_{4} {\lambda}_{1}^{2}}{2} - \frac{- {d}_{1} - 2 {d}_{2} {\lambda}_{1} - 3 {d}_{3} {\lambda}_{1}^{2} - 4 {d}_{4} {\lambda}_{1}^{3}}{4 \left({\gamma}_{j} - {\lambda}_{1}\right)}, \notag \\
\wp'{\left({\mu}_{j}- z_{0}\right)} &= - \frac{\left({b}_{0} + {b}_{1} {\gamma}_{j} + {b}_{2} {\gamma}_{j}^{2}\right) \left({d}_{1} + 2 {d}_{2} {\lambda}_{1} + 3 {d}_{3} {\lambda}_{1}^{2} + 4 {d}_{4} {\lambda}_{1}^{3}\right)}{4 \left({\gamma}_{j} - {\lambda}_{1}\right)^{2}}. \label{eq:wp-points}
\end{align}

The solutions for modal powers are then:
\begin{align}
u_j{\left(z\right)} v_j{\left(z\right)} &= \rho(\mu_j) - \rho(z), \notag \\
&= \frac{\wp'{\left(z_{1}\right)}}{ \sqrt{{d}_{4}} \left(\wp{\left(z_{1}\right)} - \wp{\left({\mu}_{j} - z_{0}\right)}\right)} - \frac{\wp'{\left(z_{1}\right)}}{\sqrt{{d}_{4}} \left(\wp{\left(z_{1}\right)} - \wp{\left(z - z_{0}\right)}\right) }, \notag \\
&= \frac{\wp'{\left(z_{1}\right)} }{ \sqrt{{d}_{4}} \left(\wp{\left(z_{1}\right)} - \wp{\left({\mu}_{j}- z_{0}\right)}\right)} \frac{ \left(\wp{\left(z - z_{0}\right)} - \wp{\left({\mu}_{j} - z_{0}\right)}\right) }{ \left(\wp{\left(z - z_{0}\right)} - \wp{\left(z_{1}\right)}\right) } \label{eq:uv-wp}
\end{align}


% ---------- SECTION -------------
% --------------------------------

\section{Solutions for modes in terms of Weierstrass \texorpdfstring{$\sigma$}{sigma}, \texorpdfstring{$\zeta$}{zeta} functions}

Through substitution of \eqref{eq:ham-uv}, \eqref{eq:duv_j}, \eqref{eq:uvj-gam-rho}, and \eqref{eq:power-conserved} into \eqref{eq:uv-system}, it can be shown that \eqref{eq:uv-system} can be written as logarithmic derivatives such that the right hand side is a function of $\rho$ and its derivative $\rho'$ in the form:
\begin{align}
\frac{\frac{\partial}{\partial z} u_j{\left(z\right)}}{u_j{\left(z\right)}} &= \frac{1}{2}\frac{\rho'{\left(z \right)} - \rho'{\left({\mu}_{j} \right)}}{\rho{\left(z \right)} - \rho{\left({\mu}_{j} \right)}} + \rho{\left(z \right)} {\Lambda}_{1,j} + {\Lambda}_{0,j}, \notag \\
\frac{\frac{\partial}{\partial z} v_j{\left(z\right)}}{v_j{\left(z\right)}} &= \frac{1}{2}\frac{\rho'{\left(z \right)} + \rho'{\left({\mu}_{j} \right)}}{\rho{\left(z \right)} - \rho{\left({\mu}_{j} \right)}} - \rho{\left(z \right)} {\Lambda}_{1,j} - {\Lambda}_{0,j}, \label{eq:dlog-u-v-rho} \\
{\Lambda}_{0,j} &= - {a}_{j} - \frac{{\gamma}_{j}}{4} \sum\limits_{k,l=1}^{4} {a}_{k,l} - \sum_{k=1}^{4} {a}_{j,k} {\gamma}_{k} + \frac{1}{2}\sum_{k=1}^{4} {\gamma}_{k} \sum_{l=1}^{4} {a}_{k,l} + \frac{1}{2}\sum_{k=1}^{4} {a}_{k}, \\
{\Lambda}_{1,j} &= \sum_{k=1}^{4} {a}_{j,k} - \frac{1}{4}\sum\limits_{k,l=1}^{4}  {a}_{k,l}.
\end{align}

We substitute \eqref{eq:uv-wp} into \eqref{eq:dlog-u-v-rho} and use the elliptic function identity:
\begin{equation}
\frac{\wp'{\left(x,g_{2},g_{3} \right)}}{\wp{\left(x,g_{2},g_{3} \right)} - \wp{\left(y,g_{2},g_{3} \right)}} = \zeta{\left(x + y,g_{2},g_{3} \right)} + \zeta{\left(x - y,g_{2},g_{3} \right)} - 2 \zeta{\left(x,g_{2},g_{3} \right)}
\end{equation}
to write the right hand side in terms of $\zeta$ and constants:
\begin{align}
\frac{\frac{\partial}{\partial z} u{\left(z,{\mu}_{j} \right)}}{u{\left(z,{\mu}_{j} \right)}} =& \frac{\left(\zeta{\left(z - z_{0} + z_{1} \right)} - 2 \zeta{\left(z_{1} \right)} - \zeta{\left(z - z_{0} - z_{1} \right)}\right) {\Lambda}_{1,j}}{\sqrt{{d}_{4}}} \nonumber \\[6pt]
& + \zeta{\left(z - 2 z_{0} + {\mu}_{j} \right)} - \frac{\zeta{\left(z - z_{0} - z_{1} \right)}}{2} - \frac{\zeta{\left(z - z_{0} + z_{1} \right)}}{2} \nonumber \\[6pt]
& - \frac{\zeta{\left({\mu}_{j} - z_{0} - z_{1}\right)}}{2} - \frac{\zeta{\left({\mu}_{j} - z_{0} + z_{1}\right)}}{2} + {\Lambda}_{0,j} + {\Lambda}_{1,j} {\lambda}_{1} \notag \\[12pt]
\frac{\frac{\partial}{\partial z} v{\left(z,{\mu}_{j} \right)}}{v{\left(z,{\mu}_{j} \right)}} =& - \frac{\left(\zeta{\left(z - z_{0} + z_{1} \right)} - 2 \zeta{\left(z_{1} \right)} - \zeta{\left(z - z_{0} - z_{1} \right)}\right) {\Lambda}_{1,j}}{\sqrt{{d}_{4}}} \nonumber \\[6pt]
& + \zeta{\left(z - {\mu}_{j} \right)} - \frac{\zeta{\left(z - z_{0} - z_{1} \right)}}{2} - \frac{\zeta{\left(z - z_{0} + z_{1} \right)}}{2} \nonumber \\[6pt]
& + \frac{\zeta{\left({\mu}_{j} - z_{0} - z_{1}\right)}}{2} + \frac{\zeta{\left({\mu}_{j} - z_{0} + z_{1} \right)}}{2} - {\Lambda}_{0,j} - {\Lambda}_{1,j} {\lambda}_{1} \label{eq:dlog-u-v-zeta}
\end{align}

Equations \eqref{eq:dlog-u-v-zeta} can be integrated by noting that the Weierstrass $\zeta$ function is the logarithmic derivative of the $\sigma$ function:
\begin{equation}
\label{eq:zeta-dlog_sigma}
\zeta\left(z, g_2, g_3\right)=\frac{\partial}{\partial z} \sigma\left(z, g_2, g_3\right).
\end{equation}

Performing the integration and taking exponentials gives the solutions for modes $u_j, v_j$:
\begin{align}
u_j{\left(z\right)} &= \frac{\alpha_j\,\sqrt{W_j} \,\sigma{\left(z - 2 z_{0} + {\mu}_{j} \right)} \exp\left(z {r}_{0,j} + \log{\left(\frac{\sigma{\left(z - z_{0} + z_{1} \right)}}{\sigma{\left(z - z_{0} - z_{1} \right)}} \right)} {r}_{1,j}\right)}{\sqrt{\wp{\left(z_{1} \right)} - \wp{\left(z - z_{0} \right)}} \sigma{\left({\mu}_{j} - z_{0}\right)} \sigma{\left(z - z_{0} \right)}}, \notag \\[12pt]
v_j{\left(z\right)} &= \frac{\sqrt{W_j} \, \sigma{\left(z - {\mu}_{j} \right)} \exp\left(- z {r}_{0,j} - \log{\left(\frac{\sigma{\left(z - z_{0} + z_{1} \right)}}{\sigma{\left(z - z_{0} - z_{1} \right)}} \right)} {r}_{1,j} \right)}{\alpha_j\,\sqrt{\wp{\left(z_{1} \right)} - \wp{\left(z - z_{0} \right)}} \sigma{\left({\mu}_{j} - z_{0}\right)} \sigma{\left(z - z_{0} \right)}} \label{eq:u-v-quartic}
\end{align}

where $\alpha_j$ is the integration constant that can be fixed by initial conditions to capture any phase offset between a mode and its conjugate, and where the other constants are:
\begin{align}
W_j &= \frac{\wp'{\left(z_{1} \right)}}{\left(\wp{\left(z_{1} \right)} - \wp{\left({\mu}_{j} - z_{0}\right)}\right) \sqrt{{d}_{4}}}, \\[6pt]
r_{0,j} &= {\Lambda}_{0,j} + {\Lambda}_{1,j} {\lambda}_{1} - \frac{2 \zeta{\left(z_{1} \right)} {\Lambda}_{1,j}}{\sqrt{{d}_{4}}} - \frac{\zeta{\left({\mu}_{j} - z_{0} - z_{1}\right)}}{2} - \frac{\zeta{\left({\mu}_{j} - z_{0} + z_{1}\right)}}{2}, \\[6pt]
r_{1,j} &= \frac{{\Lambda}_{1,j}}{\sqrt{{d}_{4}}}.
\end{align}



% ---------- SECTION -------------
% --------------------------------

\section{When quartic terms cancel to leave a cubic}

There will be no $\rho(z)^4$ term in the differential equation when $d_4=0$. 
For $d_4=0$, we require $b_2 \pm 2$, which is a parameter requirement, i.e., it cannot be achieved in general through tailored initial conditions of the modes.
This case is simpler to solve in some ways but it is not easily obtained by taking the limit $d_4\rightarrow0$ because $d_4$ appears in the denominator in several places. 
When $d_4=0$, we get $\wp'(z_1)=0$, i.e., $z_1$ is congruent to a half-period of the elliptic function, commonly denoted $\omega_i$, for $i=1,2,3$. 
It can then be shown either by solving the differential equation starting from the cubic, or by applying elliptic function identities to intermediate results in the quartic derivation,
that the solutions for modes simplify to:

\begin{align}
u_j{\left(z\right)} &= \frac{\alpha_j\, 2 \sigma{\left(z - 2 z_{0} + 2 z_{1} + {\mu}_{j}\right)} \exp\left( z \left( {\Lambda}_{0,j} + \frac{{d}_{2}}{3 {d}_{3}} {\Lambda}_{1,j} -\zeta{\left({\mu}_{j} -z_{0} + z_{1}\right)}\right) \right)}{ \sqrt{{d}_{3}}\, \sigma{\left({\mu}_{j} - z_{0} + z_{1}\right)} \sigma{\left(z - z_{0} + z_{1}\right)}}, \notag \\[12pt]
v_j{\left(z\right)} &= \frac{2 \sigma{\left(z - {\mu}_{j}\right)} \exp\left( - z \left({\Lambda}_{0,j} + \frac{{d}_{2}}{3 {d}_{3}}{\Lambda}_{1,j}  -\zeta{\left({\mu}_{j} -z_{0} + z_{1}\right)}\right) \right)}{ \alpha_j\, \sqrt{{d}_{3}}\, \sigma{\left({\mu}_{j} - z_{0} + z_{1}\right)} \sigma{\left(z - z_{0} + z_{1}\right)}} \label{eq:u-v-cubic}
\end{align}
where the integration constant $\alpha_j$ is to be determined by initial conditions using \eqref{eq:u-v-cubic} and is therefore not the same $\alpha_j$ as in \eqref{eq:u-v-quartic}, but where all other parameters retain their prior definitions.
The modal power simplies to:
\begin{equation}
\label{eq:modal-power-cubic}
u_j{\left(z\right)} v_j{\left(z\right)} = \frac{4 \wp{\left({\mu}_{j} - z_{0} + z_{1}\right)}}{{d}_{3}} - \frac{4 \wp{\left(z - z_{0} + z_{1}\right)}}{{d}_{3}}
\end{equation}


% ---------- SECTION -------------
% --------------------------------

\section{Transforming to canonical coordinates}

In this section we will demonstrate an application of the analytic solution by showing how its form guides us towards a canonical coordinate system for the analysis of four-wave mixing.
In the canonical coordinate system:
\begin{itemize}
    \item there is no cross-phase modulation XPM 
    \item solutions are single-valued meromorphic functions, namely Kronecker theta functions
    \item polynomials in modal powers in the differential equations for modal power are cubic not quartic
    \item conservation laws, intermodal power ratios, the cross-ratio between any four distinct modal powers, and the structure of the differential equations in the dynamic system are all preserved under the transformation                                                                                                                           
\end{itemize}

\subsection{Removing cross-phase modulation with a gauge transform}

Let us consider a gauge transform of the system in \eqref{eq:uv-system} of the following form:

\begin{align}
u_j(z) &= \hat{u}_j(z) e^{- \phi_j(z)} \notag \\
v_j(z) &= \hat{v}_j(z) e^{\phi_j(z)} \\ 
\sum_{j=1}^{4} \phi_j(z) &= 0
\end{align}

where transforming the conjugate mode with the opposing phase leaves modal powers unchanged, and where the sum over $\phi_j$ being zero ensures we do not encounter any new terms appearing in the exponents of wave mixing terms.
We make the following choice for $\phi_j$ composed of a linear part ($L$) and a nonlinear part ($NL$) (where $L$ and $NL$ are function labels not exponents):

\begin{align}
\phi_j(z) &= \phi_j^{L}(z) + \phi_j^{NL}(z) \\
\phi_j^{L}(z) &= z\left({a}_{j} - \frac{1}{4}\sum_{m=1}^{4} a_{m} - \frac{{\gamma}_{j}}{4} \sum\limits_{l,k=1}^{4} a_{l,k}\right) \\
\phi_j^{NL}(z)  &= \sum_{k=1}^{4} \left({a}_{j,k} - \frac{1}{4}\sum_{l=1}^{4} {a}_{l,k}\right) \int \hat{u}_k (z) \hat{v}_k (z) \, dz \\
\sum_{j=1}^{4} \phi_j(z) &= \sum_{j=1}^{4} \phi_j^{L}(z) = \sum_{j=1}^{4} \phi_j^{NL}(z) =0.
\end{align}

This transforms \eqref{eq:uv-system} into the following system:
\begin{align}
\frac{\partial}{\partial z} \hat{u}_j &= \left(\frac{b_1}{4} - \frac{b_2}{2}\hat{u}_j \hat{v}_j \right) \hat{u}_j + \prod\limits_{k=1, k \ne j}^{4} \hat{v}_k \notag \\
\frac{\partial}{\partial z} \hat{v}_j &= \left( - \frac{b_1}{4} + \frac{b_2}{2}\hat{u}_j \hat{v}_j\right) \hat{v}_j - \prod\limits_{k=1, k \ne j}^{4} \hat{u}_k \label{eq:uv-hat-system}
\end{align}

for which the conserved canonical Hamiltonian is:
\begin{align}
\hat{H} &= \prod_{l=1}^{4} \hat{u}_l + \prod_{l=1}^{4} \hat{v}_l - \frac{1}{4}\sum_{l=1}^{4} \left(b_2 \hat{u}^{2}_l \hat{v}^{2}_l  - b_1 \hat{u}_l \hat{v}_l \right)
\end{align}
and the intermodal power conservation laws are unchanged such that $\hat{u}_j \hat{v}_j - \hat{u}_k \hat{v}_k = \gamma_j - \gamma_k$.
In comparison to \eqref{eq:uv-system}, we see that in \eqref{eq:uv-hat-system} we now have a single phase velocity coefficient for all modes $b_1/4$, a single self-phase modulation coefficient $b_2/2$, and no cross-phase modulation terms.
In terms of the solutions to the system, the effect of the $\phi_j^{NL}$ transform is to remove the $z$ dependent log terms weighted by $r_{1,j}$ in \eqref{eq:u-v-quartic}.

\subsection{Reducing the quartic to a cubic via local normalisation}

Next, we show how we can perform a coordinate transformation that involves a local, i.e. $z$ dependent, normalisation of the modes themselves and, at the same time, induces a Mobius transform on modal powers. 
The transform is constructed such that the Mobius transform is that of the classic trick that reduces the quartic to the cubic but it is implemented at the mode level so that we can observe the corresponding effect on the coupled system.
When we do this, we will see that it further reduces the number of parameters in the system such that the phase velocity coefficient becomes the reciprocal of the SPM coefficient.
Let us define the local normalisation function $h(z)$ and its associated coordinate transform $\hat{u}_j, \hat{v}_j \rightarrow \bar{u}_j, \bar{v}_j$ (note the hat to bar change in notation):
\begin{align}
\hat{u}_j(z) &= 2 \sqrt{{\lambda}_{1} \left({\gamma}_{j} - {\lambda}_{1}\right)} \frac{\bar{u}_j(z) e^{-\theta_j z}}{\sqrt{h{\left(z \right)}}} \notag \\
\hat{v}_j(z) &= 2 \sqrt{{\lambda}_{1} \left({\gamma}_{j} - {\lambda}_{1}\right)}  \frac{\bar{v}_j(z) e^{\theta_j z} }{\sqrt{h{\left(z \right)}}} \notag \\
h{\left(z \right)} &= {d}_{5} - \sum_{l=1}^{4} \left({\gamma}_{l} - {\lambda}_{1}\right) \bar{u}_l \bar{v}_l = 4 {\lambda}_{1} \bar{u}_j \bar{v}_j - \frac{{\lambda}_{1}{d}_{5} }{{\gamma}_{j} - {\lambda}_{1}} \label{eq:hat-to-bar} \\
{\theta}_{j} &= \frac{b_0 + b_1 {\lambda}_{1} + b_2{\lambda}_{1}^2}{2\left({\gamma}_{j} - {\lambda}_{1}\right)} + \frac{{b}_{1}}{4} + \frac{{b}_{2} {\gamma}_{j}}{2} + \frac{{b}_{2} {\lambda}_{1}}{2} - \frac{1}{\chi}, \quad \sum\limits_{j=1}^{4} \theta_j = 0\\
\chi &= \frac{8\left(b_0 + b_1 {\lambda}_{1} + b_2{\lambda}_{1}^2\right)}{d_5} = \frac{16}{d_5}\sqrt{\prod\limits_{j=1}^4 \gamma_j - \lambda_1}
\end{align}

The transformation in \eqref{eq:hat-to-bar} sends \eqref{eq:uv-hat-system} to:
\begin{align}
\frac{\partial}{\partial z} \bar{u}_j &= -\left(\frac{1}{\chi} - 4\chi\,\bar{u}_j \bar{v}_j \right) \bar{u}_j - 4\chi \prod\limits_{k=1, k \ne j}^{4} \bar{v}_k, \notag \\
\frac{\partial}{\partial z} \bar{v}_j &= \left( \frac{1}{\chi} - 4\chi\,\bar{u}_j \bar{v}_j\right) \bar{v}_j + 4\chi \prod\limits_{k=1, k \ne j}^{4} \bar{u}_k \label{eq:uv-bar-system}
\end{align}

for which the conserved canonical Hamiltonian is:
\begin{align}
\bar{H} &= \sum_{l=1}^{4} \left(\frac{\chi}{8} \bar{u}^{2}_l \bar{v}^{2}_l  - \frac{1}{\chi}  \bar{u}_l \bar{v}_l \right) - \frac{\chi}{4}\prod_{l=1}^{4} \bar{u}_l - \frac{\chi}{4}\prod_{l=1}^{4} \bar{v}_l.
\end{align}

To solve this system, we may define the function $\bar{w}(z)$ in terms of $h$ by:
\begin{align}
h{\left(z \right)} = \frac{\left({d}_{2} + 3 {d}_{3} {\lambda}_{1} + 6 {d}_{4} {\lambda}_{1}^{2}\right) {\lambda}_{1}}{3} - 4 \bar{w}{\left(z \right)} {\lambda}_{1}
\end{align}

and we find then that:
\begin{align}
\left(\frac{d}{d z} \bar{w}{\left(z \right)}\right)^{2} &= 4\,\bar{w}{\left(z \right)}^3 - g_2\,\bar{w}{\left(z \right)} - g_3, \notag \\
\bar{w}(z) &= \wp (z - z_0, g_2, g_3) \label{eq:dwp-bar} 
\end{align}
where $z_0$, $g_2$, and $g_3$ retain their previous definitions. 
Thus, we have seen that as a result of the transformation in \eqref{eq:hat-to-bar}, the right hand side of \eqref{eq:dwp-bar} is cubic in $\bar{w}$ and thus any differential equation in $h$ or $\bar{u}_j\bar{v}_j$ is also cubic.

\subsection{A parameterless system via rescaling}

It is now simple to remove the remaining parameter via a scaling of the modes and the length variable as follows (note the bar to tilde notation change):
\begin{align}
\bar{u}_j(z) &= -\frac{2i}{\chi}\tilde{u}_j\left(\frac{z}{\chi}\right) \\
\bar{v}_j(z) &= \frac{2i}{\chi}\tilde{v}_j\left(\frac{z}{\chi}\right) \\
z &= \chi\xi
\end{align}
This substitution transforms \eqref{eq:uv-bar-system} to:
\begin{align}
\frac{\partial}{\partial \xi} \tilde{u}_j &= -\left(1 - \tilde{u}_j \tilde{v}_j \right) \tilde{u}_j - \prod\limits_{k=1, k \ne j}^{4} \tilde{v}_k \notag \\
\frac{\partial}{\partial \xi} \tilde{v}_j &= \left( 1 - \tilde{u}_j \tilde{v}_j\right) \tilde{v}_j + \prod\limits_{k=1, k \ne j}^{4} \tilde{u}_k \label{eq:uv-tilde-system}
\end{align}

for which the conserved canonical Hamiltonian is:
\begin{align}
\tilde{H} &= \sum_{l=1}^{4} \left(\frac{1}{2} \tilde{u}^{2}_l \tilde{v}^{2}_l  - \tilde{u}_l \tilde{v}_l \right) - \prod_{l=1}^{4} \tilde{u}_l - \prod_{l=1}^{4} \tilde{v}_l
\end{align}

% ---------- SECTION -------------
% --------------------------------

\section{Plotting analytic vs numeric solutions}

To check the validity of the solutions, we evaluate the analytic solutions using Sympy in Python and we plot them against numeric solutions of the corresponding system of differential equations found using the DOP853 Runge-Kutta algorithm from SciPy in Python.


\begin{figure}[htbp]
\centering
\includegraphics[width=1.0\textwidth]{plots/case1_plot_1.pdf}
\caption{Real part of analytic Equation~\eqref{eq:uv-wp} against numeric solutions.}
\label{fig:u-v-re}
\end{figure}

\begin{figure}[htbp]
\centering
\includegraphics[width=1.0\textwidth]{plots/case1_plot_2.pdf}
\caption{Imaginary part of analytic Equation~\eqref{eq:uv-wp} against numeric solutions.}
\label{fig:u-v-im}
\end{figure}

In Figures \ref{fig:u-v-re} and \ref{fig:u-v-im} we plot the real and imaginary parts, respectively, of the product of $u_jv_j$ in an abstract non-physical $u,v$ scenario.
The analytic dashed lines are shown to be in agreement with the numeric solutions shown as symbols.

\begin{figure}[htbp]
\centering
\includegraphics[width=1.0\textwidth]{plots/case2_plot_1.pdf}
\caption{$\left|A_j\right|^2$ using analytic Equation~\eqref{eq:u-v-quartic} in \eqref{eq:u-v-A} against numeric solutions.}
\label{fig:A-abs}
\end{figure}

\begin{figure}[htbp]
\centering
\includegraphics[width=1.0\textwidth]{plots/case2_plot_2.pdf}
\caption{Phase $\phi_j$ using analytic Equation~\eqref{eq:u-v-quartic} in \eqref{eq:u-v-A} against numeric solutions.}
\label{fig:A-phi}
\end{figure}

In Figures \ref{fig:A-abs} and \ref{fig:A-phi} we plot the absolute value squared and phase, respectively, of $A_j$ in a physical four-wave mixing case.
The value of $A_j$ was found using \eqref{eq:u-v-quartic} to first obtain $u_j, v_j$, before converting to $A_j, A_j^{*}$ using \eqref{eq:u-v-A}.
The analytic dashed lines are shown to be in agreement with the numeric solutions shown as symbols. 
As expected in nonlinear fiber optic four-wave mixing, periodic intermodal power exchange is seen between modes in Figure \ref{fig:A-abs}, 
while the phase plot in Figure \ref{fig:A-phi} is dominated by linear phase velocity effects with some nonlinear effects visible as areas of high curvature in the plotted lines.

\section{Conclusion}
Summarize results and possible future directions.

\section*{Acknowledgements}
Acknowledge funding or helpful discussions.

\appendix


% ---------- APPENDIX ------------
% --------------------------------

\section{Parameter definitions}
\label{app:param-def-appendix}

\begin{align}
{b}_{0} &= {a}_{0} + \sum_{j=1}^{4} {a}_{j} {\gamma}_{j} + \frac{1}{2}\sum\limits_{j,k=1}^{4} {a}_{j,k} {\gamma}_{j} {\gamma}_{k}, \notag \\
{b}_{1} &= - \sum_{j=1}^{4} {a}_{j} - \frac{1}{2}\sum\limits_{j,k=1}^{4} \left({\gamma}_{j} + {\gamma}_{k}\right) {a}_{j,k}, \notag \\
{b}_{2} &= \frac{1}{2}\sum\limits_{j,k=1}^{4} {a}_{j,k}, \notag \\
{c}_{0} &= \prod_{j=1}^{4} {\gamma}_{j}, \notag \\
{c}_{1} &= - \sum_{k=1}^{4} \prod_{\substack{j=1 \\ j \ne k}}^{4} {\gamma}_{j}, \notag \\
{c}_{2} &= \frac{3}{2} \sum_{j=1}^{4} {\gamma}_{j}^{2} - \frac{1}{4}\sum\limits_{j,k=1}^{4} \left({\gamma}_{j} - {\gamma}_{k}\right)^{2} = - \frac{1}{2}\sum_{j=1}^{4} {\gamma}_{j}^{2}, \notag \\
{c}_{3} &= 0, \notag \\
{c}_{4} &= 1, \notag \\
{d}_{0} &= {b}_{0}^{2} - 4 {c}_{0}, \notag \\
{d}_{1} &= 2 {b}_{0} {b}_{1} - 4 {c}_{1}, \notag \\
{d}_{2} &= 2 {b}_{0} {b}_{2} + {b}_{1}^{2} - 4 {c}_{2}, \notag \\
{d}_{3} &= 2 {b}_{1} {b}_{2} - 4 {c}_{3}, \notag \\
{d}_{3} &= 2 {b}_{1} {b}_{2} - 4 {c}_{3}, \notag \\
{d}_{4} &= {b}_{2}^{2} - 4 {c}_{4}
\end{align}


% ---------- APPENDIX ------------
% --------------------------------

\section{Initial condition relations}
\label{app:init-conds}

\begin{align}
\rho{\left(0 \right)} &= - \frac{1}{4}\sum_{j=1}^{4} u_j{\left(0 \right)} v_j{\left(0 \right)}, \notag \\
{\gamma}_{j} &= u_j{\left(0\right)} v_j{\left(0 \right)} - \frac{1}{4}\sum_{j=1}^{4} u_j{\left(0\right)} v_j{\left(0\right)}, \notag \\
\left. \frac{d}{d z} \rho{\left(z \right)} \right|_{\substack{ z=0 }} &= \prod_{j=1}^{4} u_j(0) - \prod_{j=1}^{4} v_j(0)
\end{align}


% ---------- APPENDIX ------------
% --------------------------------

\section{The Frobenius-Stickelberger determinant}
\label{app:fs-det-ham-proof}

The Frobenius-Stickelberger (FS) determinant formula is an elliptic function identity that relates products of $\sigma$ functions to a multivariate determinant of higher order derivatives of $\wp$.
In this section we show that the conservation of the Hamiltonian can be viewed as a manifestation of FS.
We are interested in the $4\times4$ case of FS in the form:
\begin{align}
\label{eq:fs-det-m}
\frac{C\,\sigma{\left(z + {\nu}_{1}\right)} \sigma{\left(z + {\nu}_{2}\right)} \sigma{\left(z + {\nu}_{3}\right)} \sigma{\left(z - {\nu}_{1} - {\nu}_{2} - {\nu}_{3}\right)}}{\sigma^{4}{\left(z\right)}} & = \det M
\end{align}
where:
\begin{align}
C = &\frac{12  \sigma{\left({\nu}_{1} - {\nu}_{2}\right)} \sigma{\left({\nu}_{1} - {\nu}_{3}\right)} \sigma{\left({\nu}_{2} - {\nu}_{3}\right)} }{\sigma{\left({\nu}_{1}\right)} \sigma^{4}{\left({\nu}_{2}\right)} \sigma^{4}{\left({\nu}_{3}\right)}} \\[6pt]
\det M = &
\begin{vmatrix}
1 & \wp(z) & \frac{\partial \wp(z)}{\partial z} & \frac{\partial^2 \wp(z)}{\partial z^2} \\
1 & \wp(\nu_1) & -\frac{\partial \wp(\nu_1)}{\partial \nu_1} & \frac{\partial^2 \wp(\nu_1)}{\partial \nu_1^2} \\
1 & \wp(\nu_2) & -\frac{\partial \wp(\nu_2)}{\partial \nu_2} & \frac{\partial^2 \wp(\nu_2)}{\partial \nu_2^2} \\
1 & \wp(\nu_3) & -\frac{\partial \wp(\nu_3)}{\partial \nu_3} & \frac{\partial^2 \wp(\nu_3)}{\partial \nu_3^2} \\
\end{vmatrix}  \notag \\[6pt]
= & - 6 \left(\wp{\left({\nu}_{1}\right)} - \wp{\left(z\right)}\right) \left(\wp{\left({\nu}_{2}\right)} - \wp{\left(z\right)}\right) \left(\wp{\left({\nu}_{1}\right)} - \wp{\left({\nu}_{2}\right)}\right) \wp'{\left({\nu}_{3}\right)} \notag \\
& + 6 \left(\wp{\left({\nu}_{1}\right)} - \wp{\left(z\right)}\right) \left(\wp{\left({\nu}_{3}\right)} - \wp{\left(z\right)}\right) \left(\wp{\left({\nu}_{1}\right)} - \wp{\left({\nu}_{3}\right)}\right) \wp'{\left({\nu}_{2}\right)}  \notag \\
& - 6 \left(\wp{\left({\nu}_{2}\right)} - \wp{\left(z\right)}\right) \left(\wp{\left({\nu}_{3}\right)} - \wp{\left(z\right)}\right) \left(\wp{\left({\nu}_{2}\right)} - \wp{\left({\nu}_{3}\right)}\right) \wp'{\left({\nu}_{1}\right)}  \notag \\
& - 6 \left(\wp{\left({\nu}_{1}\right)} - \wp{\left({\nu}_{2}\right)}\right) \left(\wp{\left({\nu}_{1}\right)} - \wp{\left({\nu}_{3}\right)}\right) \left(\wp{\left({\nu}_{2}\right)} - \wp{\left({\nu}_{3}\right)}\right) \wp'{\left(z\right)}
\end{align}

The starting point for us is the following relation established via \eqref{eq:ham-uv}, \eqref{eq:duv_j}, and \eqref{eq:Q-uv}:
\begin{align}
2 \prod_{j=1}^{4} u_j &= - \frac{\partial}{\partial z} u_j\, v_j + {a}_{0} + \sum_{j=1}^{4} u_j v_j {a}_{j} + \frac{1}{2}\sum\limits_{j,k=1}^{4} u_j v_j u_k v_k \,{a}_{j,k}, \notag \\
&=\rho' + \sum_{l=0}^{2}b_l\,\rho\left(z\right)^l. \label{eq:fs-det-start}
\end{align}
We will show how \eqref{eq:fs-det-start} can be transformed into \eqref{eq:fs-det-m}

\subsection{The left hand side}
From \eqref{eq:uv-wp}, the right hand side of \eqref{eq:fs-det-start} is a doubly periodic elliptic function,
so too then is the left. Thus for integers $m,n$ and half-periods $\omega_1, \omega_3$ we must have:
\begin{align}
\prod_{j=1}^{4} \frac{u_j{\left(z\right)}}{u_j{\left(2 m {\omega}_{3} + 2 n {\omega}_{1} + z\right)}} & = 1. \label{eq:u-prod-is-one}
\end{align}

We substitute our solution \eqref{eq:u-v-quartic} into \eqref{eq:u-prod-is-one} and use the quasi-periodicity of $\sigma$:
\begin{align}
\sigma{\left(2 m {\omega}_{3} + 2 n {\omega}_{1} + z\right)} = \left(-1\right)^{m n + m + n} \sigma{\left(z\right)} e^{\left(2 m {\omega}_{3} + 2 n {\omega}_{1} + 2 z\right) \zeta{\left(m {\omega}_{3} + n {\omega}_{1}\right)}} \label{eq:quasi-sigma}
\end{align}

to reduce the product in \eqref{eq:u-prod-is-one} to an exponential which, as it is equal to one, must have argument $2\pi i \mathrm{N}(n,m)$, with $\mathrm{N}(n,m)$ an integer that varies with $n,m$.
This leads to:
\begin{align}
2 i \pi \mathrm{N}{\left(n,m \right)} &= - \left(2 m {\omega}_{3} + 2 n {\omega}_{1}\right) \sum_{j=1}^{4} {r}_{0,j} - 2 \zeta{\left(m {\omega}_{3} + n {\omega}_{1}\right)} \sum_{j=1}^{4} {\nu}_{j}, \label{eq:exp-arg-is-one} \\ 
\sum_{j=1}^{4} {\nu}_{j} &= 2 \mathrm{N}{\left(1,0 \right)} {\omega}_{3} - 2 \mathrm{N}{\left(0,1 \right)} {\omega}_{1} = 0 \pmod{\text{lattice}}, \label{eq:nuj-sum-zero}  \\ 
\sum_{j=1}^{4} {r}_{0,j} &= 2 \,\zeta{\left(\mathrm{N}{\left(0,1 \right)} {\omega}_{1} - \mathrm{N}{\left(1,0 \right)} {\omega}_{3}\right)} \label{eq:roj-is-zeta} 
\end{align}

where $\nu_j=\mu_j-z_0$, and in moving from \eqref{eq:exp-arg-is-one} to \eqref{eq:nuj-sum-zero} and \eqref{eq:roj-is-zeta} we made use of the identity $\zeta{\left({\omega}_{3}\right)} {\omega}_{1} - \zeta{\left({\omega}_{1}\right)} {\omega}_{3} = i \pi /2$. 
Ultimately, equations \eqref{eq:quasi-sigma}, \eqref{eq:nuj-sum-zero} and \eqref{eq:roj-is-zeta} allow us to make the following substitution in the left hand side of \eqref{eq:fs-det-start}:
\begin{align}
\frac{\sigma{\left(z + {\nu}_{4} \right)}}{\sigma{\left({\nu}_{4} \right)}}\,\exp{\left(z \sum\limits_{j=1}^{4} {r}_{0,j}\right)} = - \frac{\sigma{\left(z - {\nu}_{1} - {\nu}_{2} - {\nu}_{3} \right)}}{\sigma{\left({\nu}_{1} + {\nu}_{2} + {\nu}_{3} \right)}}
\end{align}

\subsection{The right hand side}
The following algebraic identity follows from Waring-Lagrange interpolation and holds for any values of the variables and functions:
\begin{align}
\sum_{l=0}^{2}b_l\,\rho\left(z\right)^l =\, &\frac{\left({\gamma}_{1} - \rho{\left(z \right)}\right) \left({\gamma}_{2} - \rho{\left(z \right)}\right)}{\left({\gamma}_{1} - {\gamma}_{3}\right) \left({\gamma}_{2} - {\gamma}_{3}\right)} \sum\limits_{l=0}^{2}b_l\,\gamma_1^l + \notag \\
&\frac{\left({\gamma}_{1} - \rho{\left(z \right)}\right) \left({\gamma}_{3} - \rho{\left(z \right)}\right)}{\left({\gamma}_{1} - {\gamma}_{2}\right) \left({\gamma}_{3} - {\gamma}_{2}\right)} \sum\limits_{l=0}^{2}b_l\,\gamma_2^l + \notag \\
&\frac{\left({\gamma}_{2} - \rho{\left(z \right)}\right) \left({\gamma}_{3} - \rho{\left(z \right)}\right)}{\left({\gamma}_{1} - {\gamma}_{2}\right) \left({\gamma}_{1} - {\gamma}_{3}\right)} \sum\limits_{l=0}^{2}b_l\,\gamma_3^l \label{eq:waring-lagrange}
\end{align}

where in our case $\gamma_j=\rho(\mu_j)$ and the sum over $b_l \gamma_j$ can be written using \eqref{eq:drho-sqrd-2} as:
\begin{align}
\sum\limits_{l=0}^{2}b_l\,\gamma_j^l = \rho'\left(\mu_j\right).
\end{align}

We can the use \eqref{eq:uv-wp} to express $\rho$ in terms of $\wp$ to substitute all variables in \eqref{eq:waring-lagrange} for values in terms of $\wp, \wp'$. 
We then divide both sides of \eqref{eq:fs-det-start} by:
\begin{align}
\frac{\wp'{\left(z_{1}\right)}}{\sqrt{{d}_{4}} \left(\wp{\left(z_{1}\right)} - \wp{\left(z - z_{0}\right)}\right)^2 }
\end{align}
and observe that it takes the equivalent form to \eqref{eq:fs-det-m} up to constant factors that may look different but can be shown to be identical by multiplying by $\sigma(z)^4$ and taking $z\rightarrow 0$.

\bibliographystyle{plain}
\bibliography{references}  % Create references.bib file

\end{document}

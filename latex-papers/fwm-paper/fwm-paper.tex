\documentclass[12pt,a4paper]{article}

% Packages for math, symbols, and formatting
\usepackage{amsmath, amssymb, amsfonts}  % Standard math packages
\usepackage{graphicx}                     % For figures
\usepackage{hyperref}                     % For links
\usepackage{geometry}                     % Page layout
\usepackage{caption}                      % Captions for figures/tables
\usepackage{bm}                           % Bold math symbols

% Page layout
\geometry{top=2.5cm, bottom=2.5cm, left=2.5cm, right=2.5cm}

% Numbering equations by section
\numberwithin{equation}{section}

% Title and authors
\title{The general analytic solution to continuous wave four-wave mixing in nonlinear fiber optics}
\author{Graham Hesketh \
\small United Kingdom \
\small \texttt{[gdh1e10@gmail.com](mailto:gdh1e10@gmail.com)}}
\date{\today}

\begin{document}

\maketitle

\begin{abstract}
The general analytic solution to continuous wave four-wave mixing in nonlinear optical fibers is presented in terms of Weierstrass elliptic $\wp, \sigma, \zeta$ functions. 
Solutions are provided for the full complex envelopes for all four frequency modes, under all initial conditions, and without any undepleted pump approximation.
\end{abstract}

\section{Introduction}
Four-wave mixing (FWM) in nonlinear optical fibers is one of the fundamental parametric processes enabled by the Kerr nonlinearity. 
It underpins a wide range of applications, including wavelength conversion, parametric amplification, frequency-comb generation, and quantum light sources. 
Despite this broad relevance, obtaining closed-form analytic descriptions of FWM remains challenging because the underlying coupled-wave equations are nonlinear, phase-sensitive, and generally require numerical integration. 
As a result, most textbook treatments rely on simplifying assumptions such as undepleted pumps, negligible phase mismatch, or weak signal and idler powers.

Several works have presented analytic or semi-analytic solutions in specific regimes, including the undepleted-pump limit, the perfectly phase-matched case, or configurations with constrained input conditions. 
However, a fully general analytic solution—valid for arbitrary pump depletion, arbitrary phase mismatch, and arbitrary input power ratios—remains of significant theoretical and practical interest. 
Such a solution not only clarifies the structure of the FWM interaction but also provides a unified benchmark against which approximate models and numerical simulations can be evaluated.

In this work, we derive the general analytic solution for continuous-wave four-wave mixing in a nonlinear optical fiber. 
Beginning from the standard coupled-wave equations for the interacting fields, we identify the conserved quantities associated with the parametric interaction and use them to reduce the system to an integrable form. 
The resulting expressions describe the full evolution of the pump, signal, and idler amplitudes, including amplitude and phase dynamics, for arbitrary initial conditions. 
These solutions recover the known limiting cases and offer direct physical insight into gain behavior, conversion efficiency, and phase evolution across the full parameter space.

\section{The continuous wave four-wave mixing system}
The following coupled system of ordinary differential equations is taken from Agrawal and describes four-wave mixing in the continuous wave limit, i.e., in the absence of time derivatives:

\begin{align}
\label{eq:fwm-system}
\frac{dA_1}{dz} &= \frac{in_2\omega_1}{c}\left[\left(f_{11}|A_1|^2 + 2\sum_{k\neq 1} f_{1k}|A_k|^2\right)A_1 + 2f_{1234}A_2^*A_3A_4e^{i\Delta kz}\right], \\
\frac{dA_2}{dz} &= \frac{in_2\omega_2}{c}\left[\left(f_{22}|A_2|^2 + 2\sum_{k\neq 2} f_{2k}|A_k|^2\right)A_2 + 2f_{2134}A_1^*A_3A_4e^{i\Delta kz}\right], \\
\frac{dA_3}{dz} &= \frac{in_2\omega_3}{c}\left[\left(f_{33}|A_3|^2 + 2\sum_{k\neq 3} f_{3k}|A_k|^2\right)A_3 + 2f_{3412}A_1A_2A_4^*e^{-i\Delta kz}\right], \\
\frac{dA_4}{dz} &= \frac{in_2\omega_4}{c}\left[\left(f_{44}|A_4|^2 + 2\sum_{k\neq 4} f_{4k}|A_k|^2\right)A_4 + 2f_{4312}A_1A_2A_3^*e^{-i\Delta kz}\right].
\end{align}

where:
\begin{itemize}
    \item $A_j$ are the slowly-varying complex field amplitudes for waves $j = 1, 2, 3, 4$
    \item $z$ is the propagation distance
    \item $\omega_j$ are the angular frequencies of the respective waves
    \item $c$ is the speed of light in vacuum
    \item $n_2$ is the nonlinear refractive index
    \item $f_{jj}$ are the self-phase modulation (SPM) coefficients
    \item $f_{jk}$ (for $j \neq k$) are the cross-phase modulation (XPM) coefficients
    \item $f_{jklm}$ are the four-wave mixing (FWM) coefficients
    \item $A^*$ denotes complex conjugation
    \item $\Delta k = \beta_1 + \beta_2 - \beta_3 - \beta_4$ is the phase mismatch
    \item $\beta_j = n(\omega_j)\omega_j/c$ are the propagation constants
\end{itemize}

The modal overlap integrals are defined in terms of the transverse distribution of the fiber mode $F_j(x,y)$ as:
\begin{align}
\label{eq:overlaps}
f_{jklm} &= \frac{\langle F_j^* F_k^* F_l F_m \rangle}{\sqrt{\langle |F_j|^2 \rangle \langle |F_k|^2 \rangle \langle |F_l|^2 \rangle \langle |F_m|^2 \rangle}}, \\
f_{jk} &= f_{jjkk}, \\
f_{jj} &= f_{jjjj}, 
\end{align}
where $\langle \cdots \rangle = \iint_{-\infty}^{\infty} (\cdots) \, dx\,dy$ denotes the transverse spatial integral.

Agrawal says of \eqref{eq:fwm-system} that the equations ``are quite general in the sense that they include the effects of SPM, XPM, and pump depletion on the FWM process; a numerical approach is necessary to solve them exactly.''
That said, herein, they are solved analytically in full, as written, without any further approximation.
\section{Results}
Describe your findings. Include figures if needed:

\begin{figure}[h]
\centering
\includegraphics[width=0.6\textwidth]{plots/case1_plot_1.pdf}
\caption{Example figure caption.}
\label{fig:example}
\end{figure}

\section{Conclusion}
Summarize results and possible future directions.

\section*{Acknowledgements}
Acknowledge funding or helpful discussions.

\bibliographystyle{plain}
\bibliography{references}  % Create references.bib file

\end{document}

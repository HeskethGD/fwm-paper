\documentclass[12pt,a4paper]{article}

% Packages for math, symbols, and formatting
\usepackage{amsmath, amssymb, amsfonts}  % Standard math packages
\usepackage{graphicx}                     % For figures
\usepackage{hyperref}                     % For links
\usepackage{geometry}                     % Page layout
\usepackage{caption}                      % Captions for figures/tables
\usepackage{bm}                           % Bold math symbols

% Page layout
\geometry{top=2.5cm, bottom=2.5cm, left=2.5cm, right=2.5cm}

% Numbering equations by section
\numberwithin{equation}{section}

% Title and authors
\title{The general analytic solution to continuous wave four-wave mixing in nonlinear fiber optics in Weierstra\ss{} notation}
\author{Graham Hesketh \
\small United Kingdom \
\small \texttt{[gdh1e10@gmail.com](mailto:gdh1e10@gmail.com)}}
\date{\today}

\begin{document}

\maketitle

\begin{abstract}
The general analytic solution to continuous wave four-wave mixing in nonlinear optical fibers is presented in terms of Weierstra\ss{} elliptic $\wp, \sigma, \zeta$ functions. 
Solutions are provided for the full complex envelopes for all four frequency modes, under all initial conditions, and without any undepleted pump approximation.
\end{abstract}


% ---------- SECTION -------------
% --------------------------------

\section{Introduction}
Four-wave mixing (FWM) in nonlinear optical fibers is one of the fundamental parametric processes enabled by the Kerr nonlinearity. 
It underpins a wide range of applications, including wavelength conversion, parametric amplification, frequency-comb generation, and quantum light sources. 
Despite this broad relevance, obtaining closed-form analytic descriptions of FWM remains challenging because the underlying coupled-wave equations are nonlinear, phase-sensitive, and generally require numerical integration. 
As a result, most textbook treatments rely on simplifying assumptions such as undepleted pumps, negligible phase mismatch, or weak signal and idler powers.

Several works have presented analytic or semi-analytic solutions in specific regimes, including the undepleted-pump limit, the perfectly phase-matched case, or configurations with constrained input conditions. 
However, a fully general analytic solution—valid for arbitrary pump depletion, arbitrary phase mismatch, and arbitrary input power ratios—remains of significant theoretical and practical interest. 
Such a solution not only clarifies the structure of the FWM interaction but also provides a unified benchmark against which approximate models and numerical simulations can be evaluated.

In this work, we derive the general analytic solution for continuous-wave four-wave mixing in a nonlinear optical fiber. 
Beginning from the standard coupled-wave equations for the interacting fields, we identify the conserved quantities associated with the parametric interaction and use them to reduce the system to an integrable form. 
The resulting expressions describe the full evolution of the pump, signal, and idler amplitudes, including amplitude and phase dynamics, for arbitrary initial conditions. 
These solutions recover the known limiting cases and offer direct physical insight into gain behavior, conversion efficiency, and phase evolution across the full parameter space.


% ---------- SECTION -------------
% --------------------------------

\section{The continuous wave four-wave mixing system}
The following coupled system of ordinary differential equations is taken from Agrawal and describes four-wave mixing in the continuous wave limit, i.e., in the absence of time derivatives:

\begin{align}
\label{eq:fwm-system}
\frac{dA_1}{dz} &= \frac{in_2\omega_1}{c}\left[\left(f_{11}|A_1|^2 + 2\sum_{k\neq 1} f_{1k}|A_k|^2\right)A_1 + 2f_{1234}A_2^*A_3A_4e^{i\Delta kz}\right], \\
\frac{dA_2}{dz} &= \frac{in_2\omega_2}{c}\left[\left(f_{22}|A_2|^2 + 2\sum_{k\neq 2} f_{2k}|A_k|^2\right)A_2 + 2f_{2134}A_1^*A_3A_4e^{i\Delta kz}\right], \\
\frac{dA_3}{dz} &= \frac{in_2\omega_3}{c}\left[\left(f_{33}|A_3|^2 + 2\sum_{k\neq 3} f_{3k}|A_k|^2\right)A_3 + 2f_{3412}A_1A_2A_4^*e^{-i\Delta kz}\right], \\
\frac{dA_4}{dz} &= \frac{in_2\omega_4}{c}\left[\left(f_{44}|A_4|^2 + 2\sum_{k\neq 4} f_{4k}|A_k|^2\right)A_4 + 2f_{4312}A_1A_2A_3^*e^{-i\Delta kz}\right].
\end{align}

where:
\begin{itemize}
    \item $A_j$ are the slowly-varying complex field amplitudes for waves $j = 1, 2, 3, 4$
    \item $z$ is the propagation distance
    \item $\omega_j$ are the angular frequencies of the respective waves
    \item $c$ is the speed of light in vacuum
    \item $n_2$ is the nonlinear refractive index
    \item $f_{jj}$ are the self-phase modulation (SPM) coefficients
    \item $f_{jk}$ (for $j \neq k$) are the cross-phase modulation (XPM) coefficients
    \item $f_{jklm}$ are the four-wave mixing (FWM) coefficients
    \item $A^*$ denotes complex conjugation
    \item $\Delta k = \beta_1 + \beta_2 - \beta_3 - \beta_4$ is the phase mismatch
    \item $\beta_j = n(\omega_j)\omega_j/c$ are the propagation constants
\end{itemize}

The modal overlap integrals are defined in terms of the transverse distribution of the fiber mode $F_j(x,y)$ as:
\begin{align}
\label{eq:overlaps}
f_{jklm} &= \frac{\langle F_j^* F_k^* F_l F_m \rangle}{\sqrt{\langle |F_j|^2 \rangle \langle |F_k|^2 \rangle \langle |F_l|^2 \rangle \langle |F_m|^2 \rangle}}, \\
f_{jk} &= f_{kj} = f_{jjkk}
\end{align}
where $\langle \cdots \rangle = \iint_{-\infty}^{\infty} (\cdots) \, dx\,dy$ denotes the transverse spatial integral.

Agrawal says of \eqref{eq:fwm-system} that the equations ``are quite general in the sense that they include the effects of SPM, XPM, and pump depletion on the FWM process; a numerical approach is necessary to solve them exactly.''
That said, herein, they are solved analytically in full, as written, without any further approximation.


% ---------- SECTION -------------
% --------------------------------

\section{Simplifying parameter dependence}
From \eqref{eq:overlaps}, we observe the following symmetries among the wave mixing coefficients:

\begin{align}
\label{eq:overlap-symms}
f_{1234} &= |f_{1234}|e^{i\theta}, \\
f_{2134} &= f_{1234} = |f_{1234}|e^{i\theta}, \\
f_{3412} &= f_{1234}^* = |f_{1234}|e^{-i\theta}, \\
f_{4312} &= f_{1234}^* = |f_{1234}|e^{-i\theta},
\end{align}
where $\theta$ is the phase of $f_{1234}$. 
Consequently, we can conveniently renormalise the functions, and also absorb the phase $\theta$ as a global phase rotation of the modes, 
in such a way that the wave mixing coefficients all become equal to one.
To do so, we introduce the following redefinition of the mode functions 
(take care to note that, while not crucial, the complex conjugates are shared among $u,v$ in this choice of labels so as to later conveniently give one product over $u$ and one over $v$):

\begin{align}
T &= \sqrt{\frac{2\,n_2 \left|f_{1234}\right|}{c} \sqrt{ \prod\limits_{k=1}^{4} \omega_k}}, \\
u_1\left(z\right) &= \frac{T\,\,e^{-i\pi/4}\,e^{-i\theta/4}}{\sqrt{\omega_{1}}} \,A_{1}\left(z\right)\,e^{i z \beta_{1}}, \\
u_2\left(z\right) &= \frac{T\,e^{-i\pi/4}\,e^{-i\theta/4}}{\sqrt{\omega_{2}}}\,A_{2}\left(z\right)\,e^{i z \beta_{2}}, \\
u_3\left(z\right) &= \frac{T\,e^{i\pi/4}\,e^{-i\theta/4}}{\sqrt{\omega_{3}}}\,A^*_{3}\left(z\right)\,e^{-i z \beta_{3}}, \\
u_4\left(z\right) &= \frac{T\,e^{i\pi/4}\,e^{-i\theta/4}}{\sqrt{\omega_{4}}}\,A^*_{4}\left(z\right)\,e^{-i z \beta_{4}}, \\
v_1\left(z\right) &= \frac{T\,e^{-i\pi/4}\,e^{i\theta/4}}{\sqrt{\omega_{1}}}\,A^*_{1}\left(z\right)\,e^{-i z \beta_{1}}, \\
v_2\left(z\right) &= \frac{T\,e^{-i\pi/4}\,e^{i\theta/4}}{\sqrt{\omega_{2}}}\,A^*_{2}\left(z\right)\,e^{-i z \beta_{2}}, \\
v_3\left(z\right) &= \frac{T\,e^{i\pi/4}\,e^{i\theta/4}}{\sqrt{\omega_{3}}}\,A_{3}\left(z\right)\,e^{i z \beta_{3}}, \\
v_4\left(z\right) &= \frac{T\,e^{i\pi/4}\,e^{i\theta/4}}{\sqrt{\omega_{4}}}\,A_{4}\left(z\right)\,e^{i z \beta_{4}}.
\end{align}

We subsequently redefine the remaining phase modulation parameters as:
\begin{align}
{a}_{j} &= - i\, s{\left(j \right)} {\beta}_{j}, \\
{a}_{j,k} &= {a}_{k,j} = - \frac{\left(\delta_{j k} - 2\right) s{\left(j \right)}\, s{\left(k \right)}\, {f}_{j,k}\, {\omega}_{j}\, {\omega}_{k}}{2 \left|{{f}_{1,2,3,4}}\right| \sqrt{\prod_{l=1}^{4} {\omega}_{l}}}
\end{align}
with Kronecker $\delta$, and $s(j)$ defined such that $s(1) = s(2) = 1, s(3) = s(4) = -1$.

The four-wave mixing system in \eqref{eq:fwm-system} is thus recast as:
\begin{align}
\label{eq:uv-system}
\frac{d}{d z} u_j{\left(z \right)} &= - \left({a}_{j} + \sum_{k=1}^{4} {a}_{j,k}\,u_k v_k \right) u_j + \prod\limits_{k=1, k \ne j}^{4} v_k, \\
\frac{d}{d z} v_j{\left(z \right)} &= \left({a}_{j} + \sum_{k=1}^{4} {a}_{j,k}\,u_k v_k \right) v_j - \prod\limits_{k=1, k \ne j}^{4} u_k
\end{align}


% ---------- SECTION -------------
% --------------------------------

\section{Conserved quantities}

Based on prior experience of similar systems we are prompted to ask if the system in \eqref{eq:uv-system} is a canonical Hamiltonian system, and indeed we confirm that it is with:
\begin{align}
\label{eq:ham-uv}
H(u_1,\dots, u_4, v_1, \dots, v_4) &= -\sum_{j=1}^{4} a_{j}\, u_j v_j - \frac{1}{2}\,\sum_{j,k=1}^{4} a_{j,k}\, u_j v_j u_k v_k + \prod\limits_{j=1}^{4} u_j + \prod\limits_{j=1}^{4} v_j, \\
\frac{d}{d z} u_j{\left(z \right)} &= \frac{\partial H}{\partial v_j}, \\
\frac{d}{d z} v_j{\left(z \right)} &= -\frac{\partial H}{\partial u_j}, \\
\frac{d}{d z} H &= 0
\end{align}

The conservation of $H$ is to be expected as a consequence of \eqref{eq:ham-uv} lacking an explicit $z$ dependence. The pair $u_j, v_j$ are Hamiltonian conjugates, and each pair represents one of four degrees of freedom in the system. 
We refer to their product $u_j v_j$ as the modal power and it evolves according to:
\begin{equation}
\label{eq:duv_j}
\frac{d}{d z} u_j v_j = \prod\limits_{k=1}^{4} v_k - \prod\limits_{k=1}^{4} u_k
\end{equation}

As the right hand side of \eqref{eq:duv_j} is the same for all $j$, 
we may define constants $\gamma_j$ and function $\rho(z)$ such that:
\begin{equation}
\label{eq:uvj-gam-rho}
u_j(z) v_j(z) = \gamma_j - \rho(z)
\end{equation}

from which it follows that there are 3 (read as $j > k$ to avoid over counting) intermodal power conservation laws of the form:
\begin{equation}
\label{eq:power-conserved}
u_j(z) v_j(z) - u_k(z) v_k(z) = \gamma_j - \gamma_k.
\end{equation}

The system has 4 degrees of freedom and four conserved quantities which is a requirement for integrability in the Liouville-Arnold sense.
The $\gamma_j$ constants can be determined from \eqref{eq:power-conserved} and initial conditions, 
however, this provides three equations for four unknowns and thus a choice is available for normalisation. 
We choose to impose the constraint that:
\begin{equation}
\sum\limits_{j=1}^{4}\gamma_j = 0.
\end{equation} 


% ---------- SECTION -------------
% --------------------------------

\section{Solutions for modal powers in terms of Weierstra\ss{} \texorpdfstring{$\wp$}{wp} elliptic functions}

One key thing to note in order to obtain elliptic functions solutions is that the derivative of the modal power in \eqref{eq:duv_j} is proportional to the difference of the two wave mixing product terms, and that the Hamiltonian in \eqref{eq:ham-uv} contains their sum.
We then note the simple but important identity:
\begin{equation}
\label{eq:prod-id}
\left(\prod\limits_{k=1}^{4} v_k - \prod\limits_{k=1}^{4} u_k\right)^2 - \left(\prod\limits_{k=1}^{4} v_k + \prod\limits_{k=1}^{4} u_k\right)^2 = - 4 \prod\limits_{j=1}^{4} u_j v_j
\end{equation}
which enables us to square \eqref{eq:duv_j} and replace wave mixing terms with phase modulation terms.

To proceed in this manner let us introduce the function $Q$ which represents the phase modulation part of the Hamiltonian:
\begin{equation}
\label{eq:Q-uv}
Q(u_1(z)v_1(z), \dots, u_4(z)v_4(z)) = a_0 + \sum_{j=1}^{4} a_{j}\, u_j v_j + \frac{1}{2}\,\sum_{j,k=1}^{4} a_{j,k}\, u_j v_j u_k v_k 
\end{equation}
with $a_0=H$, and observe that by squaring \eqref{eq:duv_j} and substituting \eqref{eq:ham-uv}, \eqref{eq:uvj-gam-rho}, \eqref{eq:prod-id}, and \eqref{eq:Q-uv}, we obtain:
\begin{align}
\label{eq:drho-sqrd}
\left(\frac{d}{d z} \rho{\left(z \right)}\right)^{2} &= Q^{2}{\left({\gamma}_{1} - \rho{\left(z \right)}, \dots, {\gamma}_{4}  - \rho{\left(z \right)}\right)} - 4 \prod_{j=1}^{4} \left({\gamma}_{j} - \rho{\left(z \right)}\right), \\
\left(\frac{d}{d z} \rho{\left(z \right)}\right)^{2} &= \sum_{l=0}^{4}d_l\,\rho\left(z\right)^l, \\
\left(\frac{d}{d z} \rho{\left(z \right)}\right)^{2} &= d_4 \prod_{l=1}^{4} \left(\rho\left(z\right) -  \lambda_l\right),
\end{align}
where the simple but verbose formulas for $d_l$ in terms of other parameters and initial conditions are given in the Appendix, and where $\lambda_l$ are the roots of the corresponding quartic polynomial in $\rho(z)$ such that $\sum_{l=0}^{4}d_l\,\lambda^l=0$.
We now transform \eqref{eq:drho-sqrd} from quartic to the standard cubic form of the Weierstra\ss{} $\wp$ function using the classical trick which can be conceptualised in three steps:
\begin{align}
\label{eq:wp-4-to-3}
\rho\left(z\right) &= q\left(z\right) + \lambda_1 & \text{shift so rhs is 0 for $q(z)=0$}, \\
s\left(z\right) &= 1/q\left(z\right) & \text{invert to make the quartic a cubic}, \\
w\left(z\right) &= C_1\,s\left(z\right) + C_0 & \text{shift/scale for some $C_j$ to match Weierstra\ss{} coefficients}.
\end{align}

The procedure sketched in \eqref{eq:wp-4-to-3} is implemented in the following single transformation:
\begin{align}
\label{eq:rho-wp}
\rho{\left(z \right)} &= {\lambda}_{1} + \frac{d_4}{- 4 w{\left(z \right)} \prod_{l=1}^{3} {\Omega}_{l}  + \frac{d_4}{3} \sum_{l=1}^{3} {\Omega}_{l} }, \\
\Omega_l &= \frac{1}{\lambda_{l+1} - \lambda_1},\\
\label{eq:dwp}
\left(\frac{d}{d z} w{\left(z \right)}\right)^{2} &= 4\,w{\left(z \right)}^3 - g_2\,w{\left(z \right)} - g_3, \\
g_{2} &= {d}_{0} {d}_{4} - \frac{{d}_{1} {d}_{3}}{4} + \frac{{d}_{2}^{2}}{12}, \\
g_{3} &= \frac{{d}_{0} {d}_{2} {d}_{4}}{6} - \frac{{d}_{0} {d}_{3}^{2}}{16} - \frac{{d}_{1}^{2} {d}_{4}}{16} + \frac{{d}_{1} {d}_{2} {d}_{3}}{48} - \frac{{d}_{2}^{3}}{216}
\end{align}
where the constants $g_2$ and $g_3$ are known as Weierstra\ss{} elliptic invariants. 
Equation \eqref{eq:dwp} defines the Weierstra\ss{} elliptic $\wp$ function and the solution is:
\begin{equation}
w(z) = \wp (z - z_0, g_2, g_3)
\end{equation}
where $z_0$ is a constant chosen to match initial conditions. This constant $z_0$, and others that we will now introduce, can be obtained by inverting $\wp$ using an elliptic integral, (e.g. Carlson's symmetric $R_F$ integral). 
As $\wp$ is an even function, it is neccessary to also specify a corresponding condition for the derivative when inverting, i.e., to find $z$ from known $x,y$ we give conditions such as $\wp\left(z\right)=x, \wp'\left(z\right)=y$, where $\wp'$ is the derivative of $\wp$ known as Weiertra\ss{} p-prime. 
The points obtained during such an inversion are determined modulo the period lattice of the doubly periodic $\wp$. 
Let us proceed to define the point $z_0$, and introduce the point $z_1$ as the pole of $\rho(z)$, and points $\mu_j - z_0$ as the zeroes of $u_j(z)v_j(z)$:
\begin{align}
\wp{\left(z_{0}\right)} &= \frac{{d}_{2}}{12} + \frac{{d}_{3} {\lambda}_{1}}{4} + \frac{{d}_{4} {\lambda}_{1}^{2}}{2} + \frac{- {d}_{1} - 2 {d}_{2} {\lambda}_{1} - 3 {d}_{3} {\lambda}_{1}^{2} - 4 {d}_{4} {\lambda}_{1}^{3}}{4 \left(- \rho{\left(0 \right)} + {\lambda}_{1}\right)}, \\
\wp'{\left(z_{0}\right)} &= \frac{\left({d}_{1} + 2 {d}_{2} {\lambda}_{1} + 3 {d}_{3} {\lambda}_{1}^{2} + 4 {d}_{4} {\lambda}_{1}^{3}\right) }{4 \left(\rho{\left(0 \right)} - {\lambda}_{1}\right)^{2}}\left. \frac{d}{d z} \rho{\left(z \right)} \right|_{\substack{ z=0 }} , \\
\wp{\left(z_{1}\right)} &= \frac{{d}_{2}}{12} + \frac{{d}_{3} {\lambda}_{1}}{4} + \frac{{d}_{4} {\lambda}_{1}^{2}}{2}, \\
\wp'{\left(z_{1}\right)} &= \frac{\left(- {d}_{1} - 2 {d}_{2} {\lambda}_{1} - 3 {d}_{3} {\lambda}_{1}^{2} - 4 {d}_{4} {\lambda}_{1}^{3}\right) \sqrt{{d}_{4}}}{4},\\
\wp{\left({\mu}_{j} - z_{0}\right)} &= \frac{{d}_{2}}{12} + \frac{{d}_{3} {\lambda}_{1}}{4} + \frac{{d}_{4} {\lambda}_{1}^{2}}{2} - \frac{- {d}_{1} - 2 {d}_{2} {\lambda}_{1} - 3 {d}_{3} {\lambda}_{1}^{2} - 4 {d}_{4} {\lambda}_{1}^{3}}{4 \left({\gamma}_{j} - {\lambda}_{1}\right)}, \\
\wp'{\left({\mu}_{j}- z_{0}\right)} &= - \frac{\left({b}_{0} + {b}_{1} {\gamma}_{j} + {b}_{2} {\gamma}_{j}^{2}\right) \left({d}_{1} + 2 {d}_{2} {\lambda}_{1} + 3 {d}_{3} {\lambda}_{1}^{2} + 4 {d}_{4} {\lambda}_{1}^{3}\right)}{4 \left({\gamma}_{j} - {\lambda}_{1}\right)^{2}}.
\end{align}

The solutions for modal powers are then:
\begin{align}
\label{eq:uv-wp}
u_j{\left(z\right)} v_j{\left(z\right)} &= \rho(\mu_j) - \rho(z),\\
&= \frac{\wp'{\left(z_{1}\right)}}{ \sqrt{{d}_{4}} \left(\wp{\left(z_{1}\right)} - \wp{\left({\mu}_{j} - z_{0}\right)}\right)} - \frac{\wp'{\left(z_{1}\right)}}{\sqrt{{d}_{4}} \left(\wp{\left(z_{1}\right)} - \wp{\left(z - z_{0}\right)}\right) }, \\
&= \frac{\wp'{\left(z_{1}\right)} }{ \sqrt{{d}_{4}} \left(\wp{\left(z_{1}\right)} - \wp{\left({\mu}_{j}- z_{0}\right)}\right)} \frac{ \left(\wp{\left(z - z_{0}\right)} - \wp{\left({\mu}_{j} - z_{0}\right)}\right) }{ \left(\wp{\left(z - z_{0}\right)} - \wp{\left(z_{1}\right)}\right) }
\end{align}

% ---------- SECTION -------------
% --------------------------------

\section{Solutions for modes in terms of Weierstra\ss{} \texorpdfstring{$\sigma$}{sigma}, \texorpdfstring{$\zeta$}{zeta} functions}

Through substitution of \eqref{eq:ham-uv}, \eqref{eq:duv_j}, \eqref{eq:uvj-gam-rho}, and \eqref{eq:power-conserved} into \eqref{eq:uv-system}, it can be shown that \eqref{eq:uv-system} can be written as logarithmic derivatives such that the right hand side is a function of $\rho$ and its derivative $\rho'$ in the form:
\begin{align}
\label{eq:dlog-u-v-rho}
\frac{\frac{\partial}{\partial z} u_j{\left(z\right)}}{u_j{\left(z\right)}} &= \frac{1}{2}\frac{\rho'{\left(z \right)} - \rho'{\left({\mu}_{j} \right)}}{\rho{\left(z \right)} - \rho{\left({\mu}_{j} \right)}} + \rho{\left(z \right)} {\Lambda}_{1,j} + {\Lambda}_{0,j}, \\
\frac{\frac{\partial}{\partial z} v_j{\left(z\right)}}{v_j{\left(z\right)}} &= \frac{1}{2}\frac{\rho'{\left(z \right)} + \rho'{\left({\mu}_{j} \right)}}{\rho{\left(z \right)} - \rho{\left({\mu}_{j} \right)}} - \rho{\left(z \right)} {\Lambda}_{1,j} - {\Lambda}_{0,j}, \\
{\Lambda}_{0,j} &= - {a}_{j} - \frac{{\gamma}_{j}}{4} \sum\limits_{k,l=1}^{4} {a}_{k,l} - \sum_{k=1}^{4} {a}_{j,k} {\gamma}_{k} + \frac{1}{2}\sum_{k=1}^{4} {\gamma}_{k} \sum_{l=1}^{4} {a}_{k,l} + \frac{1}{2}\sum_{k=1}^{4} {a}_{k}, \\
{\Lambda}_{1,j} &= \sum_{k=1}^{4} {a}_{j,k} - \frac{1}{4}\sum\limits_{k,l=1}^{4}  {a}_{k,l}.
\end{align}

We can then substitute \eqref{eq:uv-wp} into \eqref{eq:dlog-u-v-rho} and make use of the following Weierstra\ss{} elliptic function identity:
\begin{equation}
\frac{\wp'{\left(x,g_{2},g_{3} \right)}}{\wp{\left(x,g_{2},g_{3} \right)} - \wp{\left(y,g_{2},g_{3} \right)}} = \zeta{\left(x + y,g_{2},g_{3} \right)} + \zeta{\left(x - y,g_{2},g_{3} \right)} - 2 \zeta{\left(x,g_{2},g_{3} \right)}
\end{equation}
which gives:
\begin{align}
\label{eq:dlog-u-v-zeta}
\frac{\frac{\partial}{\partial z} u{\left(z,{\mu}_{j} \right)}}{u{\left(z,{\mu}_{j} \right)}} =& \frac{\left(\zeta{\left(z - z_{0} + z_{1} \right)} - 2 \zeta{\left(z_{1} \right)} - \zeta{\left(z - z_{0} - z_{1} \right)}\right) {\Lambda}_{1,j}}{\sqrt{{d}_{4}}} \nonumber \\[6pt]
& + \zeta{\left(z - 2 z_{0} + {\mu}_{j} \right)} - \frac{\zeta{\left(z - z_{0} - z_{1} \right)}}{2} - \frac{\zeta{\left(z - z_{0} + z_{1} \right)}}{2} \nonumber \\[6pt]
& - \frac{\zeta{\left({\mu}_{j} - z_{0} - z_{1}\right)}}{2} - \frac{\zeta{\left({\mu}_{j} - z_{0} + z_{1}\right)}}{2} + {\Lambda}_{0,j} + {\Lambda}_{1,j} {\lambda}_{1} \\[12pt]
\frac{\frac{\partial}{\partial z} v{\left(z,{\mu}_{j} \right)}}{v{\left(z,{\mu}_{j} \right)}} =& - \frac{\left(\zeta{\left(z - z_{0} + z_{1} \right)} - 2 \zeta{\left(z_{1} \right)} - \zeta{\left(z - z_{0} - z_{1} \right)}\right) {\Lambda}_{1,j}}{\sqrt{{d}_{4}}} \nonumber \\[6pt]
& + \zeta{\left(z - {\mu}_{j} \right)} - \frac{\zeta{\left(z - z_{0} - z_{1} \right)}}{2} - \frac{\zeta{\left(z - z_{0} + z_{1} \right)}}{2} \nonumber \\[6pt]
& + \frac{\zeta{\left({\mu}_{j} - z_{0} - z_{1}\right)}}{2} + \frac{\zeta{\left({\mu}_{j} - z_{0} + z_{1} \right)}}{2} - {\Lambda}_{0,j} - {\Lambda}_{1,j} {\lambda}_{1}
\end{align}

Equations \eqref{eq:dlog-u-v-zeta} can be integrated by noting that the Weierstra\ss{} $\zeta$ function is the logarithmic derivative of the Weierstra\ss{} $\sigma$ function:
\begin{equation}
\label{eq:zeta-dlog_sigma}
\zeta\left(z, g_2, g_3\right)=\frac{\partial}{\partial z} \sigma\left(z, g_2, g_3\right).
\end{equation}

Performing the integration and taking exponentials gives the solutions for modes $u_j, v_j$ as:
% \begin{align}
% \label{eq:u-v-j-sol}
% u_j{\left(z\right)} &= \frac{{\alpha}_{j} \sqrt{\wp'{\left(z_{1} \right)}} \sigma{\left(z_{1} \right)} \sigma{\left(z - 2 z_{0} + {\mu}_{j} \right)} \exp\left(z {r}_{0,j} + \log{\left(\frac{\sigma{\left(z - z_{0} + z_{1} \right)}}{\sigma{\left(z - z_{0} - z_{1} \right)}} \right)} {r}_{2,j}\right)}{\sqrt[4]{{d}_{4}}\sqrt{\wp{\left(z_{1} \right)} - \wp{\left({\mu}_{j} - z_{0}\right)}} \sigma{\left({\mu}_{j} - z_{0}\right)} \sigma{\left(z - z_{0} - z_{1} \right)}} 
% \end{align}


\begin{align}
u{\left(z\right)} &= \frac{\alpha_j\,\sqrt{W_j} \,\sigma{\left(z - 2 z_{0} + {\mu}_{j} \right)} \exp\left(z {r}_{0,j} + \log{\left(\frac{\sigma{\left(z - z_{0} + z_{1} \right)}}{\sigma{\left(z - z_{0} - z_{1} \right)}} \right)} {r}_{1,j}\right)}{\sqrt{\wp{\left(z_{1} \right)} - \wp{\left(z - z_{0} \right)}} \sigma{\left({\mu}_{j} - z_{0}\right)} \sigma{\left(z - z_{0} \right)}}, \\[12pt]
v_j{\left(z\right)} &= \frac{\sqrt{W_j} \, \sigma{\left(z - {\mu}_{j} \right)} \exp\left(- z {r}_{0,j} - \log{\left(\frac{\sigma{\left(z - z_{0} + z_{1} \right)}}{\sigma{\left(z - z_{0} - z_{1} \right)}} \right)} {r}_{1,j} \right)}{\alpha_j\,\sqrt{\wp{\left(z_{1} \right)} - \wp{\left(z - z_{0} \right)}} \sigma{\left({\mu}_{j} - z_{0}\right)} \sigma{\left(z - z_{0} \right)}}
\end{align}

where $\alpha_j$ is the integration constant that can be fixed by initial conditions to capture any phase offset between a mode and its conjugate, and where the other constants are:
\begin{align}
W_j &= \frac{\wp'{\left(z_{1} \right)}}{\left(\wp{\left(z_{1} \right)} - \wp{\left({\mu}_{j} - z_{0}\right)}\right) \sqrt{{d}_{4}}}, \\[6pt]
r_{0,j} &= {\Lambda}_{0,j} + {\Lambda}_{1,j} {\lambda}_{1} - \frac{2 \zeta{\left(z_{1} \right)} {\Lambda}_{1,j}}{\sqrt{{d}_{4}}} - \frac{\zeta{\left({\mu}_{j} - z_{0} - z_{1}\right)}}{2} - \frac{\zeta{\left({\mu}_{j} - z_{0} + z_{1}\right)}}{2}, \\[6pt]
r_{1,j} &= \frac{{\Lambda}_{1,j}}{\sqrt{{d}_{4}}}.
\end{align}


\begin{figure}[h]
\centering
\includegraphics[width=0.6\textwidth]{plots/case1_plot_1.pdf}
\caption{Example figure caption.}
\label{fig:example}
\end{figure}

\section{Conclusion}
Summarize results and possible future directions.

\section*{Acknowledgements}
Acknowledge funding or helpful discussions.

\bibliographystyle{plain}
\bibliography{references}  % Create references.bib file

\end{document}

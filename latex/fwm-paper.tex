\newif\ifpreprint
% \preprinttrue
\preprintfalse

\ifpreprint
  \documentclass[12pt,a4paper]{article}
\else
  \documentclass{iopjournal}
\fi

% Packages for math, symbols, and formatting
\usepackage{fix-cm}                       % Allow arbitrary font sizes
\usepackage{amsmath, amssymb, amsfonts}  % Standard math packages
\usepackage{graphicx}                     % For figures
\usepackage{hyperref}                     % For links
\usepackage{caption}                      % Captions for figures/tables
\usepackage{bm}                           % Bold math symbols
\ifpreprint
  \usepackage{geometry}                     % Page layout
  \geometry{top=2.5cm, bottom=2.5cm, left=2.5cm, right=2.5cm}
\fi

% Add bibilography
\usepackage[backend=biber,style=numeric, sorting=none]{biblatex}
\addbibresource{CW-FWM-Analytic-Solutions.bib}


% Fix plot positions
\usepackage{float}

% Numbering equations by section
\numberwithin{equation}{section}

\begin{document}

\ifpreprint
  % Preprint format (article class)
  \title{Complete Weierstrass elliptic function solutions and canonical coordinates for four-wave mixing in nonlinear optical fibres}
  \author{Graham Hesketh \\
  \small \texttt{gdh1e10@gmail.com}}
  \date{\today}
  \maketitle
\else
  % Journal format (iopjournal class)
  % \articletype{Paper}
  \title{Complete Weierstrass elliptic function solutions and canonical coordinates for four-wave mixing in nonlinear optical fibres}
  
  \author{Graham Hesketh}

  \email{gdh1e10@gmail.com}

  \keywords{Four-wave mixing, Weierstrass elliptic functions, Nonlinear fibre optics, Kronecker theta functions, Integrable systems}
\fi

\begin{abstract}
The general analytic solution to quasi-continuous-wave four-wave mixing in nonlinear optical fibres is presented in terms of Weierstrass elliptic $\wp, \sigma, \zeta$ functions. 
Solutions provide the full complex envelopes for all four frequencies, under all initial conditions, without undepleted pump approximation. 
A canonical coordinate system is identified in which solutions simplify to Kronecker theta functions. The Hamiltonian is shown to be a manifestation of the Frobenius-Stickelberger determinant. Numerical validation is shown.
\end{abstract}


% ---------- SECTION -------------
% --------------------------------

\section{Introduction}
Four-wave mixing (FWM) in nonlinear optical fibres is one of the fundamental parametric processes enabled by the Kerr nonlinearity. 
In FWM, nonlinear interactions enable energy exchange between four optical waves whose frequencies satisfy specific phase-matching conditions, allowing the conversion of photons from pump frequencies to signal and idler frequencies \cite{agrawalNonlinearFiberOptics2019}.
It underpins a wide range of applications, including wavelength conversion, parametric amplification, ultrafast signal processing, frequency-comb generation, and quantum light sources (see \cite{agrawalNonlinearFiberOptics2019} and references therein for applications). 
Despite this broad relevance, obtaining closed-form analytic descriptions of FWM remains challenging because the underlying coupled-wave equations are nonlinear and phase-sensitive. 
As a result, most textbook treatments rely on simplifying assumptions such as undepleted pumps, identical effective mode areas, negligible phase mismatch, or weak signal and idler powers.

Two notable bodies of work have previously presented relevant analytic solutions for four-wave mixing. 
Firstly, in \cite{chenFourphotonParametricMixing1989, chenFourwaveMixingOptical1989} solutions for the power in each of the frequency modes is derived by inverting elliptic integrals to obtain elliptic functions.
Secondly, in \cite{marhicAnalyticSolutionsPhases2013} elliptic integrals are given for the phases for each mode. 
Both works use Jacobi elliptic functions and treat amplitude and phase separately.
In contrast, the work herein derives complete solutions for the complex mode envelopes themselves, avoiding decomposition into amplitude and phase or real and imaginary parts, which provides a more direct description that naturally preserves all phase relationships and may therefore provide benefits for coherent applications.

The solutions derived herein are expressed using Weierstrass elliptic functions rather than the Jacobi elliptic functions employed in prior work.
The Weierstrass formulation provides a natural solution pathway for complex envelopes: the coupled equations are recast as logarithmic derivatives, which directly yield Weierstrass $\zeta$ functions, and integration then produces solutions in terms of Weierstrass $\sigma$ functions.
This progression through the Weierstrass function hierarchy—from $\wp$ to $\zeta$ to $\sigma$—emerges organically from the structure of the complex envelope equations, whereas Jacobi elliptic functions are more naturally suited to systems where amplitudes and phases are treated separately.
Moreover, the connection to Hamiltonian structure and conservation laws is more transparent in Weierstrass notation.
In particular, we show that the Hamiltonian conservation arises from the Frobenius--Stickelberger determinant formula, an elliptic function identity most naturally expressed using Weierstrass functions \cite{whittakerCourseModernAnalysis2021, frobeniusZurTheorieElliptischen1877}.
This identification may facilitate future generalizations to higher-order mixing processes or multimode systems and furthermore, this connection reveals that Hamiltonian conservation in four-wave mixing is not merely a convenient construct but rather a manifestation of deep structural properties of elliptic functions.
The derivation of these complete analytic solutions in Weierstrass form is the first of the two main results of this paper.

The structure of the resulting analytic solutions naturally motivates a transformation to a new set of canonical coordinates. 
Remarkably, although this transformation depends explicitly on the propagation variable $z$, it leaves the coupled differential equations structurally invariant, preserving the characteristic interaction terms of four-wave mixing dynamics. 
This invariance property, which to the author's knowledge has not been previously reported for four-wave mixing, suggests deep underlying structure that may extend to other parametric processes.
At the same time, the transformation leads to a substantial simplification: the solutions become single-valued meromorphic functions, specifically Kronecker theta functions, and the system parameters are greatly reduced or eliminated entirely.
The identification of solutions as Kronecker theta functions opens the door to leveraging known Fourier series and multipole expansions of these functions, which may prove useful for further analysis of four-wave mixing dynamics.
Solutions in Kronecker theta form appear in other integrable nonlinear optical systems such as wave mixing in quadratic media and polarisation dynamics \cite{heskethGeneralComplexEnvelope2015}, pointing towards a unified mathematical description of parametric processes.
The canonical coordinates with Kronecker theta function solutions is the second of this paper's main results.

% ---------- SECTION -------------
% --------------------------------

\section{The quasi-continuous-wave four-wave mixing system}
In this section we recall the four-wave mixing system from nonlinear fiber optics that forms the focus of our study. 
The coupled system of ordinary differential equations in \eqref{eq:fwm-system} is taken from \cite{agrawalNonlinearFiberOptics2019} which references \cite{stolenParametricAmplificationFrequency1982} as the origin of the theory. 
The system describes four-wave mixing in the quasi-continuous-wave limit, i.e., when the temporal envelope variations are negligible compared to the spatial evolution along the propagation direction, allowing the partial time derivatives $\partial A_j/\partial t$ to be neglected.
Physically, the equations describe the evolution of four copropagating optical waves that exchange energy through the Kerr nonlinearity.
Each wave experiences self-phase modulation (SPM), where its own intensity induces a phase shift on itself, cross-phase modulation (XPM), where the intensities of other waves induce additional phase shifts, and parametric coupling through the four-wave mixing terms, which enable direct energy transfer between the waves when phase-matching conditions are satisfied:
\begin{align}
\frac{dA_1}{dz} &= \frac{in_2\omega_1}{c}\left[\left(f_{11}|A_1|^2 + 2\sum_{k\neq 1} f_{1k}|A_k|^2\right)A_1 + 2f_{1234}A_2^*A_3A_4e^{i\Delta kz}\right], \notag \\
\frac{dA_2}{dz} &= \frac{in_2\omega_2}{c}\left[\left(f_{22}|A_2|^2 + 2\sum_{k\neq 2} f_{2k}|A_k|^2\right)A_2 + 2f_{2134}A_1^*A_3A_4e^{i\Delta kz}\right], \notag \\
\frac{dA_3}{dz} &= \frac{in_2\omega_3}{c}\left[\left(f_{33}|A_3|^2 + 2\sum_{k\neq 3} f_{3k}|A_k|^2\right)A_3 + 2f_{3412}A_1A_2A_4^*e^{-i\Delta kz}\right], \notag \\
\frac{dA_4}{dz} &= \frac{in_2\omega_4}{c}\left[\left(f_{44}|A_4|^2 + 2\sum_{k\neq 4} f_{4k}|A_k|^2\right)A_4 + 2f_{4312}A_1A_2A_3^*e^{-i\Delta kz}\right], \label{eq:fwm-system}
\end{align}
where:
\begin{itemize}
    \item $A_j$ are the slowly-varying complex field amplitudes for waves $j = 1, 2, 3, 4$
    \item $z$ is the propagation distance
    \item $\omega_j$ are the angular frequencies of the respective waves
    \item $c$ is the speed of light in vacuum
    \item $n_2$ is the nonlinear refractive index
    \item $f_{jj}$ are the self-phase modulation (SPM) coefficients
    \item $f_{jk}$ (for $j \neq k$) are the cross-phase modulation (XPM) coefficients
    \item $f_{jklm}$ are the four-wave mixing (FWM) coefficients
    \item $A^*$ denotes complex conjugation
    \item $\Delta k = \beta_1 + \beta_2 - \beta_3 - \beta_4$ is the phase mismatch
    \item $\beta_j = n(\omega_j)\omega_j/c$ are the propagation constants.
\end{itemize}
The modal overlap integrals are defined in terms of the transverse distribution of the fiber mode $F_j(x,y)$ as:
\begin{align}
\label{eq:overlaps}
f_{jklm} &= \frac{\langle F_j^* F_k^* F_l F_m \rangle}{\sqrt{\langle |F_j|^2 \rangle \langle |F_k|^2 \rangle \langle |F_l|^2 \rangle \langle |F_m|^2 \rangle}}, \\
f_{jk} &= f_{kj} = f_{jjkk}
\end{align}
where $\langle \cdots \rangle = \iint_{-\infty}^{\infty} (\cdots) \, dx\,dy$ denotes the transverse spatial integral. Agrawal says of \eqref{eq:fwm-system} that the equations ``are quite general in the sense that they include the effects of SPM, XPM, and pump depletion on the FWM process'' and stresses that numerical or approximate solutions are commonly utilised.
Herein, they are solved analytically.


% ---------- SECTION -------------
% --------------------------------

\section{The normalised system}
In this section we normalise and generalise the system, transforming it into a form that both simplifies the algebra and broadens the scope of applicability.
From \eqref{eq:overlaps}, we observe the following symmetries among the wave mixing coefficients:

\begin{align}
f_{1234} &= |f_{1234}|e^{i\varphi}, \notag \\
f_{2134} &= f_{1234} = |f_{1234}|e^{i\varphi}, \notag \\
f_{3412} &= f_{1234}^* = |f_{1234}|e^{-i\varphi}, \notag \\
f_{4312} &= f_{1234}^* = |f_{1234}|e^{-i\varphi}, \label{eq:overlap-symms}
\end{align}
where $\varphi$ is the phase of $f_{1234}$. 
The symmetries in \eqref{eq:overlap-symms} hold exactly for scalar (weakly guiding) fiber modes with instantaneous Kerr nonlinearity, but require vectorial generalization when polarization effects or strong guidance become important.
Assuming \eqref{eq:overlap-symms}, we can conveniently renormalise the functions, and also absorb the phase $\varphi$ as a global phase rotation of the modes in such a way that the wave mixing coefficients all become equal to one.

To do so, we introduce the following redefinition of physical complex envelopes $A_j, A^*_j$ in terms of abstracted variables $u_j,v_j$.
The complex conjugates $A, A^*$ are distributed among $u,v$ in a particular pattern (note that $u_3, u_4$ involve $A^*_3, A^*_4$ while $v_3, v_4$ involve $A_3, A_4$) so as to later give one product over $u$ and one over $v$ in the wave mixing terms:

\begin{align}
T &= \sqrt{\frac{2\,n_2 \left|f_{1234}\right|}{c} \sqrt{ \prod\limits_{k=1}^{4} \omega_k}}, \label{eq:T-def} \\
u_1\left(z\right) &= \frac{T\,\,e^{-i\pi/4}\,e^{-i\varphi/4}}{\sqrt{\omega_{1}}} \,A_{1}\left(z\right)\,e^{i z \beta_{1}}, \notag \\
u_2\left(z\right) &= \frac{T\,e^{-i\pi/4}\,e^{-i\varphi/4}}{\sqrt{\omega_{2}}}\,A_{2}\left(z\right)\,e^{i z \beta_{2}}, \notag \\
u_3\left(z\right) &= \frac{T\,e^{i\pi/4}\,e^{-i\varphi/4}}{\sqrt{\omega_{3}}}\,A^*_{3}\left(z\right)\,e^{-i z \beta_{3}}, \notag \\
u_4\left(z\right) &= \frac{T\,e^{i\pi/4}\,e^{-i\varphi/4}}{\sqrt{\omega_{4}}}\,A^*_{4}\left(z\right)\,e^{-i z \beta_{4}}, \notag \\
v_1\left(z\right) &= \frac{T\,e^{-i\pi/4}\,e^{i\varphi/4}}{\sqrt{\omega_{1}}}\,A^*_{1}\left(z\right)\,e^{-i z \beta_{1}}, \notag \\
v_2\left(z\right) &= \frac{T\,e^{-i\pi/4}\,e^{i\varphi/4}}{\sqrt{\omega_{2}}}\,A^*_{2}\left(z\right)\,e^{-i z \beta_{2}}, \notag \\
v_3\left(z\right) &= \frac{T\,e^{i\pi/4}\,e^{i\varphi/4}}{\sqrt{\omega_{3}}}\,A_{3}\left(z\right)\,e^{i z \beta_{3}}, \notag \\
v_4\left(z\right) &= \frac{T\,e^{i\pi/4}\,e^{i\varphi/4}}{\sqrt{\omega_{4}}}\,A_{4}\left(z\right)\,e^{i z \beta_{4}}. \label{eq:u-v-A}
\end{align}

From this point on we do not assume that $u_j$ and $v_j$ are necessarily complex conjugates, but we will show in the following section that they are Hamiltonian conjugates with the physical four-wave mixing scenario being a particular realisation of a more general system.
In all that follows we refer to $u_j, v_j$ as modes, which we use as a general term for components of the coupled system that should not be confused with propagation modes of a multimode optical fiber. 
We refer to their product $u_j v_j$ as the modal power; while this quantity is not necessarily real-valued, the conceptual analogy is useful.

We subsequently introduce parameters $a_j, a_{j,k} \in \mathbb{C}$ which we refer to as propagation constants and phase modulation parameters respectively, again borrowing terminology by analogy.
In the four-wave mixing case these take the values:
\begin{align}
{a}_{j} &= - i\, s_j {\beta}_{j}, \\
{a}_{j,k} &= {a}_{k,j} = - \frac{\left(\delta_{j k} - 2\right) s_j\, s_k\, {f}_{j,k}\, {\omega}_{j}\, {\omega}_{k}}{2 \left|{{f}_{1,2,3,4}}\right| \sqrt{\prod_{l=1}^{4} {\omega}_{l}}}
\end{align}
with $\delta$ the Kronecker delta function, and $s_j=\pm1$ encodes the signs to match the complex conjugation among $A_j, A^*_j$ and is defined such that $s_1 = s_2 = 1, s_3 = s_4 = -1$.
The four-wave mixing system in \eqref{eq:fwm-system} is then a special case of the more general system:
\begin{align}
\frac{d}{d z} u_j{\left(z \right)} &= - \left({a}_{j} + \sum_{k=1}^{4} {a}_{j,k}\,u_k v_k \right) u_j + \prod\limits_{k=1, k \ne j}^{4} v_k, \notag \\
\frac{d}{d z} v_j{\left(z \right)} &= \left({a}_{j} + \sum_{k=1}^{4} {a}_{j,k}\,u_k v_k \right) v_j - \prod\limits_{k=1, k \ne j}^{4} u_k. \label{eq:uv-system}
\end{align}
This form encompasses the most general coupled system that the methods herein solve and may facilitate future extensions to related systems such as parity-time (PT) symmetric configurations such as the generalisation explored in \cite{heskethGeneralComplexEnvelope2015, sarmaContinuousDiscreteSchrodinger2014}.


% ---------- SECTION -------------
% --------------------------------

\section{Conserved quantities}

In this section we identify conserved quantities of the system. 
The system in \eqref{eq:uv-system} can be formulated as the following canonical Hamiltonian system:
\begin{align}
\label{eq:ham-uv}
H(u_1,\dots, u_4, v_1, \dots, v_4) &= -\sum_{j=1}^{4} a_{j}\, u_j v_j - \frac{1}{2}\,\sum_{j,k=1}^{4} a_{j,k}\, u_j v_j u_k v_k + \prod\limits_{j=1}^{4} u_j + \prod\limits_{j=1}^{4} v_j, \\
\frac{d}{d z} u_j{\left(z \right)} &= \frac{\partial H}{\partial v_j}, \\
\frac{d}{d z} v_j{\left(z \right)} &= -\frac{\partial H}{\partial u_j}, \\
\frac{d}{d z} H &= 0
\end{align}
The conservation of $H$ is to be expected as a consequence of \eqref{eq:ham-uv} lacking an explicit $z$ dependence. The pair $u_j, v_j$ are Hamiltonian conjugates, and each pair represents one of four degrees of freedom in the system. 
The modal power $u_j v_j$ evolves according to:
\begin{equation}
\label{eq:duv_j} 
\frac{d}{d z} u_j v_j = - \prod\limits_{k=1}^{4} u_k + \prod\limits_{k=1}^{4} v_k
\end{equation}

As the right hand side of \eqref{eq:duv_j} is the same for all $j$, 
we may define constants $\gamma_j$ and function $\rho(z)$ such that:
\begin{equation}
\label{eq:uvj-gam-rho}
u_j(z) v_j(z) = \gamma_j - \rho(z)
\end{equation}
from which it follows that there are 3 (read as $j > k$ to avoid over counting) intermodal power conservation laws of the form:
\begin{equation}
\label{eq:power-conserved}
u_j(z) v_j(z) - u_k(z) v_k(z) = \gamma_j - \gamma_k.
\end{equation}

The system has 4 degrees of freedom and four conserved quantities which is a requirement for integrability in the Liouville-Arnold sense \cite{arnoldMathematicalMethodsClassical1989}.
The $\gamma_j$ constants can be determined from \eqref{eq:power-conserved} and initial conditions, 
however, this provides three equations for four unknowns and thus a choice is available for normalisation. 
We choose to impose the constraint that:
\begin{equation}
\sum\limits_{j=1}^{4}\gamma_j = 0,
\end{equation} 
after which $\rho(z)$ is minus the mean modal power and $\gamma_j$ is the constant difference between a modes power and the mean.


% ---------- SECTION -------------
% --------------------------------

\section{Solutions for modal powers in terms of Weierstrass \texorpdfstring{$\wp$}{wp} elliptic functions}\label{sec:mode-power}

In this section we give analytic solutions for modal power.
The fundamental Weierstrass elliptic function theory used in this and subsequent sections can be found in \cite{whittakerCourseModernAnalysis2021}. 
In order to obtain elliptic function solutions in wave mixing dynamics, the key observation is that the derivative of the modal power in \eqref{eq:duv_j} is proportional to the difference of the two wave mixing product terms, and that the Hamiltonian in \eqref{eq:ham-uv} contains their sum.
We then note the simple but important identity:
\begin{equation}
\label{eq:prod-id}
\left(\prod\limits_{k=1}^{4} u_k - \prod\limits_{k=1}^{4} v_k\right)^2 - \left(\prod\limits_{k=1}^{4} u_k + \prod\limits_{k=1}^{4} v_k\right)^2 = - 4 \prod\limits_{j=1}^{4} u_j v_j
\end{equation}
which enables us to square \eqref{eq:duv_j} and replace wave mixing terms with phase modulation terms, i.e., monomials of $u_j v_j$ which can all be expressed in terms of $\rho(z)$ via \eqref{eq:uvj-gam-rho}.
To proceed in this manner, let us introduce the function $Q$ which represents the phase modulation part of the Hamiltonian and is thus itself expressible as a polynomial of $\rho(z)$:
\begin{align}
Q(u_1(z)v_1(z), \dots, u_4(z)v_4(z)) &= a_0 + \sum_{j=1}^{4} a_{j}\, u_j v_j + \frac{1}{2}\,\sum_{j,k=1}^{4} a_{j,k}\, u_j v_j u_k v_k , \notag  \\
& = \sum_{l=0}^{2}b_l\,\rho\left(z\right)^l \label{eq:Q-uv}
\end{align}
with $a_0=H$ a renaming that indicates the role of $H$ here as the constant term of the polynomial. Observe that squaring \eqref{eq:duv_j} and substituting \eqref{eq:ham-uv}, \eqref{eq:uvj-gam-rho}, \eqref{eq:prod-id}, and \eqref{eq:Q-uv} gives:
\begin{align}
\left(\frac{d}{d z} \rho{\left(z \right)}\right)^{2} &= Q^{2}{\left({\gamma}_{1} - \rho{\left(z \right)}, \dots, {\gamma}_{4}  - \rho{\left(z \right)}\right)} - 4 \prod_{j=1}^{4} \left({\gamma}_{j} - \rho{\left(z \right)}\right), \label{eq:drho-sqrd-1} \\
\left(\frac{d}{d z} \rho{\left(z \right)}\right)^{2} &=  \left(\sum_{l=0}^{2}b_l\,\rho\left(z\right)^l\right)^{2} - 4 \prod_{j=1}^{4} \left({\gamma}_{j} - \rho{\left(z \right)}\right), \label{eq:drho-sqrd-2} \\
\left(\frac{d}{d z} \rho{\left(z \right)}\right)^{2} &= \sum_{l=0}^{4}d_l\,\rho\left(z\right)^l, \label{eq:drho-sqrd-3} \\
\left(\frac{d}{d z} \rho{\left(z \right)}\right)^{2} &= d_4 \prod_{l=1}^{4} \left(\rho\left(z\right) -  \lambda_l\right), \label{eq:drho-sqrd-4}
\end{align}
where $b_l$, and $d_l$ are given in terms of other parameters and initial conditions in Appendices \ref{app:param-def-appendix} and \ref{app:init-conds}, and where $\lambda_l$ are the roots of the quartic polynomial in $\rho(z)$, i.e. $0=\sum_{k=0}^{4}d_k\,\lambda_{l}^k$.
Equation \eqref{eq:drho-sqrd-4} is a differential equation for an elliptic function and by separating phase modulation and wave mixing terms in the Hamiltonian, and by exploiting intermodal power conservation, we were able to obtain it directly without inverting elliptic integrals first.
To solve it, we now transform \eqref{eq:drho-sqrd-4} from quartic to the standard cubic form of the Weierstrass $\wp$ function using the classical trick which can be conceptualised in three steps \cite{whittakerCourseModernAnalysis2021}:
\begin{equation}
\label{eq:wp-4-to-3}
\begin{array}{l rcl}
\text{(1) shift by any root so RHS is 0 when $q(z)=0$} 
& \rho(z) &=& q(z) + \lambda_1,\\
\text{(2) invert to reduce the quartic to a cubic} 
& s(z) &=& \dfrac{1}{q(z)},\\
\text{(3) shift and scale to match Weierstrass coefficients} 
& w(z) &=& C_1 s(z) + C_0.
\end{array}
\end{equation}

Without loss of generality, we choose root $\lambda_1$. 
The procedure sketched in \eqref{eq:wp-4-to-3} is then implemented in the following single transformation:
\begin{align}
\rho{\left(z \right)} &= {\lambda}_{1} + \frac{d_4}{- 4 w{\left(z \right)} \prod_{l=1}^{3} {\Omega}_{l}  + \frac{d_4}{3} \sum_{l=1}^{3} {\Omega}_{l} }, \label{eq:rho-wp-1} \\
\Omega_l &= \frac{1}{\lambda_{l+1} - \lambda_1}, \notag \\
\left(\frac{d}{d z} w{\left(z \right)}\right)^{2} &= 4\,w{\left(z \right)}^3 - g_2\,w{\left(z \right)} - g_3, \label{eq:dwp} \\
g_{2} &= {d}_{0} {d}_{4} - \frac{{d}_{1} {d}_{3}}{4} + \frac{{d}_{2}^{2}}{12}, \label{eq:g2-d} \\
g_{3} &= \frac{{d}_{0} {d}_{2} {d}_{4}}{6} - \frac{{d}_{0} {d}_{3}^{2}}{16} - \frac{{d}_{1}^{2} {d}_{4}}{16} + \frac{{d}_{1} {d}_{2} {d}_{3}}{48} - \frac{{d}_{2}^{3}}{216} \label{eq:3-d}
\end{align}
where the constants $g_2$ and $g_3$ are known as Weierstrass elliptic invariants. 
Equation \eqref{eq:dwp} defines the Weierstrass elliptic $\wp$ function and the solution is:
\begin{equation}
w(z) = \wp (z - z_0, g_2, g_3)
\end{equation}
where $z_0$ is a constant chosen to match initial conditions. 
This constant $z_0$, and others that we will now introduce, can be obtained by inverting $\wp$ using an elliptic integral \cite{whittakerCourseModernAnalysis2021}. 
As $\wp$ is an even function, it is necessary to also specify a corresponding condition for the derivative when inverting to fix the sign, i.e., to find $z$ from known $x,y$ we give conditions such as $\wp\left(z\right)=x, \wp'\left(z\right)=y$, where $\wp'$ is the derivative of $\wp$ known as Weierstrass p-prime. 
From elliptic function theory, the points obtained during such an inversion are determined modulo the period lattice of the doubly periodic $\wp$, which is a two-dimensional generalisation of saying they are only uniquely determined within a period of oscillation. 
In addition to $z_0$, we introduce two other types of special points that characterize the solution structure and enable the use of elliptic function identities to solve for the modes in Section \ref{sec:mode-sols-u-v}.
The point $z_0+z_1$ corresponds to the pole of $\rho(z)$ where the denominator in \eqref{eq:rho-wp-1} vanishes, and represents a singularity in the mean modal power.
The points $\mu_j$ are the locations where individual modal powers vanish, $u_j(\mu_j)v_j(\mu_j) = 0$, and thus where $\rho(\mu_j) = \gamma_j$ from \eqref{eq:uvj-gam-rho}.
These special points are defined implicitly through their Weierstrass function values:
\begin{align}
\wp{\left(z_{0}\right)} &= \frac{{d}_{2}}{12} + \frac{{d}_{3} {\lambda}_{1}}{4} + \frac{{d}_{4} {\lambda}_{1}^{2}}{2} + \frac{- {d}_{1} - 2 {d}_{2} {\lambda}_{1} - 3 {d}_{3} {\lambda}_{1}^{2} - 4 {d}_{4} {\lambda}_{1}^{3}}{4 \left(- \rho{\left(0 \right)} + {\lambda}_{1}\right)}, \notag \\
\wp'{\left(z_{0}\right)} &= \frac{\left({d}_{1} + 2 {d}_{2} {\lambda}_{1} + 3 {d}_{3} {\lambda}_{1}^{2} + 4 {d}_{4} {\lambda}_{1}^{3}\right) }{4 \left(\rho{\left(0 \right)} - {\lambda}_{1}\right)^{2}}\left. \frac{d}{d z} \rho{\left(z \right)} \right|_{\substack{ z=0 }}, \notag \\
\wp{\left(z_{1}\right)} &= \frac{{d}_{2}}{12} + \frac{{d}_{3} {\lambda}_{1}}{4} + \frac{{d}_{4} {\lambda}_{1}^{2}}{2}, \notag \\
\wp'{\left(z_{1}\right)} &= \frac{\left(- {d}_{1} - 2 {d}_{2} {\lambda}_{1} - 3 {d}_{3} {\lambda}_{1}^{2} - 4 {d}_{4} {\lambda}_{1}^{3}\right) \sqrt{{d}_{4}}}{4}, \notag \\
\wp{\left({\mu}_{j} - z_{0}\right)} &= \frac{{d}_{2}}{12} + \frac{{d}_{3} {\lambda}_{1}}{4} + \frac{{d}_{4} {\lambda}_{1}^{2}}{2} - \frac{- {d}_{1} - 2 {d}_{2} {\lambda}_{1} - 3 {d}_{3} {\lambda}_{1}^{2} - 4 {d}_{4} {\lambda}_{1}^{3}}{4 \left({\gamma}_{j} - {\lambda}_{1}\right)}, \notag \\
\wp'{\left({\mu}_{j}- z_{0}\right)} &= - \frac{\left({b}_{0} + {b}_{1} {\gamma}_{j} + {b}_{2} {\gamma}_{j}^{2}\right) \left({d}_{1} + 2 {d}_{2} {\lambda}_{1} + 3 {d}_{3} {\lambda}_{1}^{2} + 4 {d}_{4} {\lambda}_{1}^{3}\right)}{4 \left({\gamma}_{j} - {\lambda}_{1}\right)^{2}}. \label{eq:wp-points}
\end{align}

The solutions for modal powers are then found from \eqref{eq:uvj-gam-rho} and \eqref{eq:rho-wp-1} to be:
\begin{align}
u_j{\left(z\right)} v_j{\left(z\right)} &= \rho(\mu_j) - \rho(z), \notag \\
&= \frac{\wp'{\left(z_{1}\right)}}{ \sqrt{{d}_{4}} \left(\wp{\left(z_{1}\right)} - \wp{\left({\mu}_{j} - z_{0}\right)}\right)} - \frac{\wp'{\left(z_{1}\right)}}{\sqrt{{d}_{4}} \left(\wp{\left(z_{1}\right)} - \wp{\left(z - z_{0}\right)}\right) }, \notag \\
&= \frac{\wp'{\left(z_{1}\right)} }{ \sqrt{{d}_{4}} \left(\wp{\left(z_{1}\right)} - \wp{\left({\mu}_{j}- z_{0}\right)}\right)} \frac{ \left(\wp{\left(z - z_{0}\right)} - \wp{\left({\mu}_{j} - z_{0}\right)}\right) }{ \left(\wp{\left(z - z_{0}\right)} - \wp{\left(z_{1}\right)}\right) }. \label{eq:uv-wp}
\end{align}


% ---------- SECTION -------------
% --------------------------------

\section{Solutions for modes in terms of Weierstrass \texorpdfstring{$\sigma$}{sigma}, \texorpdfstring{$\zeta$}{zeta} functions} \label{sec:mode-sols-u-v}

Having obtained the modal power solutions in the previous section, we now proceed to find solutions for the individual modes $u_j(z)$ and $v_j(z)$ themselves.
Direct substitution of \eqref{eq:ham-uv}, \eqref{eq:duv_j}, \eqref{eq:uvj-gam-rho}, and \eqref{eq:power-conserved} into \eqref{eq:uv-system} shows that the equations can be recast in terms of logarithmic derivatives:
\begin{align}
\frac{\frac{\partial}{\partial z} u_j{\left(z\right)}}{u_j{\left(z\right)}} &= \frac{1}{2}\frac{\rho'{\left(z \right)} - \rho'{\left({\mu}_{j} \right)}}{\rho{\left(z \right)} - \rho{\left({\mu}_{j} \right)}} + \rho{\left(z \right)} {\Lambda}_{1,j} + {\Lambda}_{0,j}, \notag \\
\frac{\frac{\partial}{\partial z} v_j{\left(z\right)}}{v_j{\left(z\right)}} &= \frac{1}{2}\frac{\rho'{\left(z \right)} + \rho'{\left({\mu}_{j} \right)}}{\rho{\left(z \right)} - \rho{\left({\mu}_{j} \right)}} - \rho{\left(z \right)} {\Lambda}_{1,j} - {\Lambda}_{0,j}, \label{eq:dlog-u-v-rho} \\
{\Lambda}_{0,j} &= - {a}_{j} - \frac{{\gamma}_{j}}{4} \sum\limits_{k,l=1}^{4} {a}_{k,l} - \sum_{k=1}^{4} {a}_{j,k} {\gamma}_{k} + \frac{1}{2}\sum_{k=1}^{4} {\gamma}_{k} \sum_{l=1}^{4} {a}_{k,l} + \frac{1}{2}\sum_{k=1}^{4} {a}_{k}, \\
{\Lambda}_{1,j} &= \sum_{k=1}^{4} {a}_{j,k} - \frac{1}{4}\sum\limits_{k,l=1}^{4}  {a}_{k,l},
\end{align}
where the right-hand side is expressed purely in terms of $\rho$ and its derivative $\rho'$.
This logarithmic derivative form is advantageous because it can be integrated directly to yield solutions in terms of Weierstrass $\sigma$ functions, as we now show.

We substitute the modal power solution \eqref{eq:uv-wp} into \eqref{eq:dlog-u-v-rho} and apply the elliptic function identity:
\begin{equation}
\frac{\wp'{\left(x,g_{2},g_{3} \right)}}{\wp{\left(x,g_{2},g_{3} \right)} - \wp{\left(y,g_{2},g_{3} \right)}} = \zeta{\left(x + y,g_{2},g_{3} \right)} + \zeta{\left(x - y,g_{2},g_{3} \right)} - 2 \zeta{\left(x,g_{2},g_{3} \right)}
\end{equation}
which allows us to express the right-hand side in terms of Weierstrass $\zeta$ functions and constants:
\begin{align}
\frac{\frac{\partial}{\partial z} u{\left(z,{\mu}_{j} \right)}}{u{\left(z,{\mu}_{j} \right)}} =& \frac{\left(\zeta{\left(z - z_{0} + z_{1} \right)} - 2 \zeta{\left(z_{1} \right)} - \zeta{\left(z - z_{0} - z_{1} \right)}\right) {\Lambda}_{1,j}}{\sqrt{{d}_{4}}} \nonumber \\[6pt]
& + \zeta{\left(z - 2 z_{0} + {\mu}_{j} \right)} - \frac{\zeta{\left(z - z_{0} - z_{1} \right)}}{2} - \frac{\zeta{\left(z - z_{0} + z_{1} \right)}}{2} \nonumber \\[6pt]
& - \frac{\zeta{\left({\mu}_{j} - z_{0} - z_{1}\right)}}{2} - \frac{\zeta{\left({\mu}_{j} - z_{0} + z_{1}\right)}}{2} + {\Lambda}_{0,j} + {\Lambda}_{1,j} {\lambda}_{1} \notag \\[12pt]
\frac{\frac{\partial}{\partial z} v{\left(z,{\mu}_{j} \right)}}{v{\left(z,{\mu}_{j} \right)}} =& - \frac{\left(\zeta{\left(z - z_{0} + z_{1} \right)} - 2 \zeta{\left(z_{1} \right)} - \zeta{\left(z - z_{0} - z_{1} \right)}\right) {\Lambda}_{1,j}}{\sqrt{{d}_{4}}} \nonumber \\[6pt]
& + \zeta{\left(z - {\mu}_{j} \right)} - \frac{\zeta{\left(z - z_{0} - z_{1} \right)}}{2} - \frac{\zeta{\left(z - z_{0} + z_{1} \right)}}{2} \nonumber \\[6pt]
& + \frac{\zeta{\left({\mu}_{j} - z_{0} - z_{1}\right)}}{2} + \frac{\zeta{\left({\mu}_{j} - z_{0} + z_{1} \right)}}{2} - {\Lambda}_{0,j} - {\Lambda}_{1,j} {\lambda}_{1}. \label{eq:dlog-u-v-zeta}
\end{align}
Equations \eqref{eq:dlog-u-v-zeta} can be integrated by noting that the Weierstrass $\zeta$ function is the logarithmic derivative of the Weierstrass $\sigma$ function:
\begin{equation}
\label{eq:zeta-dlog_sigma}
\zeta\left(z, g_2, g_3\right)=\frac{\partial}{\partial z} \log{\left(\sigma\left(z, g_2, g_3\right)\right)}.
\end{equation}
Performing the integration and taking exponentials gives the first main result of this paper, the complete analytic solutions for the complex mode envelopes $u_j, v_j$ without decomposition into amplitude and phase:
\begin{align}
u_j{\left(z\right)} &= \frac{\alpha_j\,\sqrt{W_j} \,\sigma{\left(z - 2 z_{0} + {\mu}_{j} \right)} \exp\left(z {r}_{0,j} + \log{\left(\frac{\sigma{\left(z - z_{0} + z_{1} \right)}}{\sigma{\left(z - z_{0} - z_{1} \right)}} \right)} {r}_{1,j}\right)}{\sqrt{\wp{\left(z_{1} \right)} - \wp{\left(z - z_{0} \right)}} \sigma{\left({\mu}_{j} - z_{0}\right)} \sigma{\left(z - z_{0} \right)}}, \notag \\[12pt]
v_j{\left(z\right)} &= \frac{\sqrt{W_j} \, \sigma{\left(z - {\mu}_{j} \right)} \exp\left(- z {r}_{0,j} - \log{\left(\frac{\sigma{\left(z - z_{0} + z_{1} \right)}}{\sigma{\left(z - z_{0} - z_{1} \right)}} \right)} {r}_{1,j} \right)}{\alpha_j\,\sqrt{\wp{\left(z_{1} \right)} - \wp{\left(z - z_{0} \right)}} \sigma{\left({\mu}_{j} - z_{0}\right)} \sigma{\left(z - z_{0} \right)}} \label{eq:u-v-quartic}
\end{align}
where the branch of the logarithm is chosen continuously along the integration path from $z=0$, $\alpha_j$ is the integration constant that can be fixed by initial conditions to capture any phase offset between a mode and its conjugate, and the other constants are:
\begin{align}
W_j &= \frac{\wp'{\left(z_{1} \right)}}{\sqrt{{d}_{4}}\,\left(\wp{\left(z_{1} \right)} - \wp{\left({\mu}_{j} - z_{0}\right)}\right)}, \\[6pt]
r_{0,j} &= {\Lambda}_{0,j} + {\Lambda}_{1,j} {\lambda}_{1} - \frac{2 \zeta{\left(z_{1} \right)} {\Lambda}_{1,j}}{\sqrt{{d}_{4}}} - \frac{\zeta{\left({\mu}_{j} - z_{0} - z_{1}\right)}}{2} - \frac{\zeta{\left({\mu}_{j} - z_{0} + z_{1}\right)}}{2}, \\[6pt]
r_{1,j} &= \frac{{\Lambda}_{1,j}}{\sqrt{{d}_{4}}}.
\end{align}
These solutions are valid for all initial conditions and include full pump depletion, cross-phase modulation, and phase mismatch effects without approximation.

As a remarkable connection, we show in Appendix~\ref{app:fs-det-ham-proof} that the Hamiltonian \eqref{eq:ham-uv} can be expressed as the Frobenius--Stickelberger determinant, a classical identity relating products of Weierstrass $\sigma$ functions to determinants of derivatives of $\wp$.
This provides a deeper structural understanding of how the Hamiltonian is conserved and connects the four-wave mixing system to the broader theory of elliptic functions.

% ---------- SECTION -------------
% --------------------------------

\section{When quartic terms cancel through special parameter values}

In this section we examine a special case in which the quartic polynomial in $\rho(z)$ reduces to a cubic, leading to simplified solutions.
This can occur when specific relationships hold among the system parameters, but it is not achievable through choice of initial conditions alone, nor is it related to the general coordinate transformation we introduce in Section \ref{sec:coord-transforms}.
Rather, this is a special instance of the original physical system \eqref{eq:fwm-system} where parameters satisfy particular constraints.

The quartic term in \eqref{eq:drho-sqrd-4} vanishes when $d_4=0$, which requires $b_2 = \pm 2$.
This is a parameter-level constraint that depends on the fiber properties and operating conditions.
When $d_4=0$, we have $\wp'(z_1)=0$, meaning that $z_1$ is congruent to a half-period of the elliptic function, commonly denoted $\omega_i$ for $i=1,2,3$.
It follows either by solving the differential equation starting from the cubic, or by applying elliptic function identities to intermediate results in the quartic derivation,
that the solutions for modes simplify to:
\begin{align}
u_j{\left(z\right)} &= \frac{\alpha_j\, 2 \sigma{\left(z - 2 z_{0} + 2 z_{1} + {\mu}_{j}\right)} \exp\left( z \left( {\Lambda}_{0,j} + \frac{{d}_{2}}{3 {d}_{3}} {\Lambda}_{1,j} -\zeta{\left({\mu}_{j} -z_{0} + z_{1}\right)}\right) \right)}{ \sqrt{{d}_{3}}\, \sigma{\left({\mu}_{j} - z_{0} + z_{1}\right)} \sigma{\left(z - z_{0} + z_{1}\right)}}, \notag \\[12pt]
v_j{\left(z\right)} &= \frac{2 \sigma{\left(z - {\mu}_{j}\right)} \exp\left( - z \left({\Lambda}_{0,j} + \frac{{d}_{2}}{3 {d}_{3}}{\Lambda}_{1,j}  -\zeta{\left({\mu}_{j} -z_{0} + z_{1}\right)}\right) \right)}{ \alpha_j\, \sqrt{{d}_{3}}\, \sigma{\left({\mu}_{j} - z_{0} + z_{1}\right)} \sigma{\left(z - z_{0} + z_{1}\right)}} \label{eq:u-v-cubic}
\end{align}
where the integration constant $\alpha_j$ is to be determined by initial conditions using \eqref{eq:u-v-cubic} and does not therefore take the same value as in \eqref{eq:u-v-quartic}, but where all other parameters retain their prior definitions.
The modal power simplies to:
\begin{equation}
\label{eq:modal-power-cubic}
u_j{\left(z\right)} v_j{\left(z\right)} = \frac{4 \wp{\left({\mu}_{j} - z_{0} + z_{1}\right)}}{{d}_{3}} - \frac{4 \wp{\left(z - z_{0} + z_{1}\right)}}{{d}_{3}}
\end{equation}
We emphasize that this is a special case arising from particular parameter values in the four-wave mixing system \eqref{eq:fwm-system}.
It should be clearly distinguished from the canonical coordinate transformation presented in Section \ref{sec:coord-transforms}, which is a general nonlinear transformation applicable to any four-wave mixing system that reduces the quartic to a cubic by transforming the variables themselves rather than requiring special parameter values.

% ---------- SECTION -------------
% --------------------------------

\section{Transforming to canonical coordinates} \label{sec:coord-transforms}

In this section we present the second main result of this paper: a canonical coordinate system for four-wave mixing that substantially simplifies both the governing equations and their solutions.
The structure of the analytic solutions in \eqref{eq:u-v-quartic} naturally suggests this transformation.
In particular, the presence of the factor $\sqrt{\wp(z_1) - \wp(z - z_0)}$ in the denominator motivates a local normalisation, while the $z$-dependent logarithmic terms weighted by $r_{1,j}$ suggest a gauge transformation to remove cross-phase modulation.

Remarkably, although the transformation depends explicitly on the propagation variable $z$, it leaves the coupled differential equations structurally invariant, preserving the characteristic four-wave mixing interaction terms while achieving substantial simplifications.
In the canonical coordinates, cross-phase modulation is eliminated, the differential equations for modal powers reduce from quartic to cubic polynomials, system parameters are greatly reduced (and can be completely removed through rescaling), and the solutions become single-valued meromorphic Kronecker theta functions.
At the same time, the transformation preserves the Hamiltonian structure, all intermodal power conservation laws, and intermodal power ratios.
We accomplish this through a sequence of three transformations, which we now present in turn.

\subsection{Removing cross-phase modulation with a gauge transform}

To begin, let us consider a transform of the system in \eqref{eq:uv-system} of the following form:
\begin{align}
u_j(z) &= \hat{u}_j(z) e^{- \phi_j(z)}, \notag \\
v_j(z) &= \hat{v}_j(z) e^{\phi_j(z)}, \\ 
\sum_{j=1}^{4} \phi_j(z) &= 0,
\end{align}
where transforming the conjugate mode with the opposing phase leaves modal powers unchanged, and where the sum over $\phi_j$ being zero ensures we do not encounter any new terms appearing in the exponents of wave mixing terms.
We make the following choice for $\phi_j$ composed of a part linear in $z$ (labelled $L$) and a part nonlinear in $z$ (labelled $NL$) (where $L$ and $NL$ are function labels not exponents):
\begin{align}
\phi_j(z) &= \phi_j^{L}(z) + \phi_j^{NL}(z) \\
\phi_j^{L}(z) &= z\left({a}_{j} - \frac{1}{4}\sum_{m=1}^{4} a_{m} - \frac{{\gamma}_{j}}{4} \sum\limits_{l,k=1}^{4} a_{l,k}\right) \\
\phi_j^{NL}(z)  &= \sum_{k=1}^{4} \left({a}_{j,k} - \frac{1}{4}\sum_{l=1}^{4} {a}_{l,k}\right) \int \hat{u}_k (z) \hat{v}_k (z) \, dz \\
\sum_{j=1}^{4} \phi_j(z) &= \sum_{j=1}^{4} \phi_j^{L}(z) = \sum_{j=1}^{4} \phi_j^{NL}(z) =0.
\end{align}

This transforms \eqref{eq:uv-system} into the following system:
\begin{align}
\frac{\partial}{\partial z} \hat{u}_j &= \left(\frac{b_1}{4} - \frac{b_2}{2}\hat{u}_j \hat{v}_j \right) \hat{u}_j + \prod\limits_{k=1, k \ne j}^{4} \hat{v}_k \notag \\
\frac{\partial}{\partial z} \hat{v}_j &= \left( - \frac{b_1}{4} + \frac{b_2}{2}\hat{u}_j \hat{v}_j\right) \hat{v}_j - \prod\limits_{k=1, k \ne j}^{4} \hat{u}_k \label{eq:uv-hat-system}
\end{align}
for which the conserved canonical Hamiltonian is:
\begin{align}
\hat{H} &= \prod_{l=1}^{4} \hat{u}_l + \prod_{l=1}^{4} \hat{v}_l - \frac{1}{4}\sum_{l=1}^{4} \left(b_2 \hat{u}^{2}_l \hat{v}^{2}_l  - b_1 \hat{u}_l \hat{v}_l \right)
\end{align}
and the intermodal power conservation laws are unchanged such that $\hat{u}_j \hat{v}_j - \hat{u}_k \hat{v}_k = \gamma_j - \gamma_k$.
Comparing \eqref{eq:uv-hat-system} with the original system \eqref{eq:uv-system}, we observe significant simplification.
All modes now share a single propagation constant $b_1/4$ and a single self-phase modulation coefficient $b_2/2$, whilst all cross-phase modulation terms $a_{j,k}$ with $j \ne k$ have been eliminated.
It should be noted that parameters $b_1, b_2$ depend on $\gamma_j$ and thus, in the new coordinates parameters depend on mode power in the original coordinates and are not exclusively fiber parameters.
In terms of the solutions to the system, the effect of the $\phi_j^{NL}$ transform is to remove the $z$ dependent log terms weighted by $r_{1,j}$ in \eqref{eq:u-v-quartic}. 
This transformation can be interpreted as an analytic form of back-propagation in that it integrates the accumulated cross-phase modulation from all modes along the propagation path and applies conjugate phase shifts to remove these effects, similar to what is implemented digitally to correct nonlinear distortion in fibre optic communication systems, or physically using phase-conjugating optical fibres as explored in \cite{heskethMinimizingInterchannelCrossphase2016}.


\subsection{Reducing the quartic to a cubic via local normalisation}\label{subsec:bar-coords}

The second transformation is a coordinate change involving a $z$-dependent normalisation of the modes, which induces a M\"obius transformation on the modal powers. 
This implements the quartic to cubic reduction discussed in Section~\ref{sec:mode-power}, but here the effect is delivered at the level of the modes $u_j, v_j$ themselves rather than only at the modal power level $u_j v_j$, allowing us to observe corresponding changes in the coupled differential equations.
The transformation further reduces the number of parameters such that the propagation constant becomes the reciprocal of the SPM coefficient.

Let us define the local normalisation function $h(z)$ and its associated coordinate transform $\left(\hat{u}_j, \hat{v}_j\right) \rightarrow \left(\bar{u}_j, \bar{v}_j \right)$ (note the hat to bar change in notation):
\begin{align}
\hat{u}_j(z) &= 2 \sqrt{{\lambda}_{1} \left({\gamma}_{j} - {\lambda}_{1}\right)} \frac{\bar{u}_j(z) e^{-\theta_j z}}{\sqrt{h{\left(z \right)}}} \notag \\
\hat{v}_j(z) &= 2 \sqrt{{\lambda}_{1} \left({\gamma}_{j} - {\lambda}_{1}\right)}  \frac{\bar{v}_j(z) e^{\theta_j z} }{\sqrt{h{\left(z \right)}}} \notag \\
h{\left(z \right)} &= {d}_{5} - \sum_{l=1}^{4} \left({\gamma}_{l} - {\lambda}_{1}\right) \bar{u}_l \bar{v}_l = 4 {\lambda}_{1} \bar{u}_j \bar{v}_j - \frac{{\lambda}_{1}{d}_{5} }{{\gamma}_{j} - {\lambda}_{1}} \label{eq:hat-to-bar}
\end{align}
where $\theta_j$ and $d_5$ are given in Appendix \ref{app:bar-coords-params}. 
The $\theta_j$ parameter is not involved in the reduction of the quartic to the cubic but its inclusion keeps the propagation constant the same for all modes after the transform.
The key parameter in the transformation is the root $\lambda_1$ previously defined in Section \ref{sec:mode-power} for the purpose of quartic to cubic reduction.
The normalisation function $h(z)$ is constructed specifically to eliminate the factor $\sqrt{\wp(z_1) - \wp(z - z_0)}$ that appears in the denominator of \eqref{eq:u-v-quartic}.
We therefore already know its effect on the solutions through the simple application in \eqref{eq:hat-to-bar}, but it is fascinating to observe the corresponding effect on the differential equations.

The transformation in \eqref{eq:hat-to-bar} sends \eqref{eq:uv-hat-system} to:
\begin{align}
\frac{\partial}{\partial z} \bar{u}_j &= -\left(\frac{1}{\chi} - \frac{\chi}{4}\bar{u}_j \bar{v}_j \right) \bar{u}_j - \frac{\varsigma \chi}{4} \prod\limits_{k=1, k \ne j}^{4} \bar{v}_k, \notag \\
\frac{\partial}{\partial z} \bar{v}_j &= \left( \frac{1}{\chi} - \frac{\chi}{4}\bar{u}_j \bar{v}_j\right) \bar{v}_j + \frac{\varsigma \chi}{4} \prod\limits_{k=1, k \ne j}^{4} \bar{u}_k \label{eq:uv-bar-system}
\end{align}
for which the conserved canonical Hamiltonian and intermodal power conservation laws are:
\begin{align}
\bar{H} &= \sum_{l=1}^{4} \left(\frac{\chi}{8} \bar{u}^{2}_l \bar{v}^{2}_l  - \frac{1}{\chi}  \bar{u}_l \bar{v}_l \right) - \frac{\varsigma \chi}{4} \left(\prod_{l=1}^{4} \bar{u}_l + \prod_{l=1}^{4} \bar{v}_l\right), \notag \\
\bar{\gamma}_j - \bar{\gamma}_k &= \bar{u}_j \bar{v}_j - \bar{u}_k \bar{v}_k,
\end{align}
with $\varsigma =\pm 1$, $\chi$ and $\bar{\gamma}_j$ given in Appendix \ref{app:bar-coords-params}.
The modal power in these coordinates evolves according to:
\begin{equation}
\label{eq:duv_bar_j} 
\frac{d}{d z} \bar{u}_j \bar{v}_j = -\frac{\varsigma \chi}{4}\left(\prod\limits_{k=1}^{4} \bar{u}_k - \prod\limits_{k=1}^{4} \bar{v}_k\right)
\end{equation}

Remarkably, the terms in \eqref{eq:uv-bar-system} are the same as those in \eqref{eq:uv-hat-system} only with different coefficients. 
We thus say that the system is invariant under the transformation, which is quite surprising given the $z$ dependence of the normalisation.
Similarly, by comparing \eqref{eq:duv_bar_j} to \eqref{eq:duv_j} we see that \eqref{eq:duv_j} is invariant up to a constant factor.

To solve this system, we may define the function $\bar{w}(z)$ in terms of $h$ by:
\begin{align}
h{\left(z \right)} = \frac{\left({d}_{2} + 3 {d}_{3} {\lambda}_{1} + 6 {d}_{4} {\lambda}_{1}^{2}\right) {\lambda}_{1}}{3} - 4 \bar{w}{\left(z \right)} {\lambda}_{1}
\end{align}
and we find then that:
\begin{align}
\left(\frac{d}{d z} \bar{w}{\left(z \right)}\right)^{2} &= 4\,\bar{w}{\left(z \right)}^3 - g_2\,\bar{w}{\left(z \right)} - g_3, \notag \\
\bar{w}(z) &= \wp (z - z_0) \label{eq:dwp-bar} \\
h(z) &= 4\lambda_1\left(\wp (z_1) - \wp (z - z_0)\right) \label{eq:dwp-h-bar} \\
\bar{u}_j(z)\bar{v}_j(z) &= \wp(\mu_j - z_0) - \wp(z - z_0)
\end{align}
where $z_0$, $z_1$, $\mu_j$, $g_2$, and $g_3$ retain their previous definitions from the original coordinates. 
We see that as a result of the transformation in \eqref{eq:hat-to-bar}, the right hand side of \eqref{eq:dwp-bar} is cubic in $\bar{w}$ and thus the corresponding differential equations for $h$ and $\bar{u}_j\bar{v}_j$ inherit this simplification as they are linearly related.
The elliptic function form of $h(z)$ in \eqref{eq:dwp-h-bar} confirms that it successfully eliminates the corresponding factor from the denominator in \eqref{eq:u-v-quartic}.

The solutions for modes $\bar{u}_j, \bar{v}_j$ are:
\begin{align}
\bar{u}_j(z) &= \frac{\bar{\alpha}_{j} \sigma{\left(z - 2 z_{0} + {\mu}_{j}\right)} \exp{\left( z \bar{\kappa}_j\right)} }{\sigma{\left(z - z_{0}\right)} \sigma{\left({\mu}_{j} - z_{0}\right)}} \notag \\
\bar{v}_j(z) &= \frac{\sigma{\left(z - {\mu}_{j}\right)} \exp{\left(- z \bar{\kappa}_j\right)}}{\bar{\alpha}_{j} \sigma{\left(z - z_{0}\right)} \sigma{\left({\mu}_{j} - z_{0}\right)} } \label{eq:u-v-bar}
\end{align}
where the Weierstrass $\sigma$ functions use the same $g_2,g_3$ from the original coordinates, 
$\bar{\kappa}_j$ is given in Appendix \ref{app:bar-coords-params}, and $\bar{\alpha}_j$ is an integration constant fixed by initial conditions and used to capture any relative phase offset between a mode $\bar{u}_j$ and its conjugate $\bar{v}_j$.

\subsection{A parameterless system via rescaling}\label{subsec:tilde-coords}

The final transformation is a simple rescaling that eliminates all remaining parameters from the system.
We scale the modes and the propagation variable as follows (note the bar to tilde notation change):
\begin{align}
\bar{u}_j(z) &= \frac{\varsigma \sqrt[4]{\varsigma}}{\chi}\tilde{u}_j(\xi), \notag \\
\bar{v}_j(z) &= \frac{\varsigma }{\sqrt[4]{\varsigma}\,\chi}\tilde{v}_j(\xi), \notag \\
z &= \chi\xi
\end{align}
where the specific form of the mode scaling is chosen to yield parameterless coefficients in the transformed equations (whilst retaining a coefficient of 4 in the cubic term of the corresponding differential equations as per the Weierstrass form).
This substitution transforms \eqref{eq:uv-bar-system} to:
\begin{align}
\frac{\partial}{\partial \xi} \tilde{u}_j &= -\left(1 - \frac{\tilde{u}_j \tilde{v}_j}{4} \right) \tilde{u}_j - \frac{1}{4}\prod\limits_{k=1, k \ne j}^{4} \tilde{v}_k, \notag \\
\frac{\partial}{\partial \xi} \tilde{v}_j &= \left( 1 - \frac{\tilde{u}_j \tilde{v}_j}{4}\right) \tilde{v}_j + \frac{1}{4}\prod\limits_{k=1, k \ne j}^{4} \tilde{u}_k, \label{eq:uv-tilde-system}
\end{align}

for which the conserved canonical Hamiltonian is:
\begin{align}
\tilde{H} &= \sum_{l=1}^{4} \left(\frac{1}{8} \tilde{u}^{2}_l \tilde{v}^{2}_l  - \tilde{u}_l \tilde{v}_l \right) - \frac{1}{4}\left(\prod_{l=1}^{4} \tilde{u}_l + \prod_{l=1}^{4} \tilde{v}_l\right), \label{eq:H-tilde}
\end{align}

and the solutions are:
\begin{align}
\tilde{u}_j{\left(\xi\right)} &= \frac{\tilde{\epsilon}_{j} \,\sigma{\left(\xi - 2 \xi_{0} + {\tilde{\mu}}_{j}\right)} \exp\left({\xi \left( 1 + \frac{{\tilde{\gamma}}_{j}}{2} - \zeta{\left( {\tilde{\mu}}_{j} - \xi_{0}\right)}\right)}\right)}{\sigma{\left(\xi - \xi_{0}\right)} \sigma{\left( {\tilde{\mu}}_{j} - \xi_{0}\right)}}, \notag \\
\tilde{v}_j{\left(\xi\right)} &= \frac{ \sigma{\left(\xi - {\tilde{\mu}}_{j}\right)} \exp\left({-\xi \left( 1 + \frac{{\tilde{\gamma}}_{j}}{2} - \zeta{\left( {\tilde{\mu}}_{j} - \xi_{0}\right)}\right)}\right)}{\tilde{\epsilon}_{j} \,\sigma{\left(\xi - \xi_{0}\right)} \sigma{\left( {\tilde{\mu}}_{j} - \xi_{0}\right)}}  \label{eq:u-v-tilde}
\end{align}
where the $\sigma$ and $\zeta$ functions use the scaled invariants $\tilde{g}_2$ and $\tilde{g}_3$ given in Appendix \ref{app:scaled-params} together with $\xi_0$, $\tilde{\epsilon}_j$, $\tilde{\mu}_j$ and $\tilde{\gamma}_j$.

The solutions in both \eqref{eq:u-v-bar} and \eqref{eq:u-v-tilde} take the form of Kronecker theta functions, which are ratios of two Weierstrass $\sigma$ functions with shifted arguments, multiplied by exponential factors.
These functions are named after Kronecker, who developed both Fourier series \cite{whittakerCourseModernAnalysis2021} and multipole expansion \cite{weilEllipticFunctionsAccording1976, charolloisEllipticFunctionsAccording2016} representations that may prove useful for further analysis in nonlinear optics.
Importantly, whilst the $\sigma$ functions and exponentials are entire functions, their ratio is a single-valued meromorphic function with well-defined poles and zeros, in contrast to the multi-valued solutions in \eqref{eq:u-v-quartic}.
The solutions are quasi-periodic, and modulo the period lattice they have zeros where the numerator $\sigma$ function vanishes and poles where the denominator $\sigma$ function vanishes.

The Kronecker theta form establishes a direct connection with other integrable nonlinear optical systems, including two-wave and three-wave mixing in quadratic nonlinear media, polarisation dynamics in nonlinear fibres, and parity-time symmetric nonlinear couplers \cite{heskethGeneralComplexEnvelope2015}, suggesting a unified mathematical framework across different parametric processes.

Owing to the parameter-free form of \eqref{eq:uv-tilde-system}, the simple structure of the Hamiltonian \eqref{eq:H-tilde}, and the elegant Kronecker theta function solutions in \eqref{eq:u-v-tilde}, we refer to $\tilde{u}_j, \tilde{v}_j$ as the canonical coordinates of four-wave mixing.
The derivation of this canonical form, together with the demonstration that it can be reached from any physical four-wave mixing system through the sequence of transformations presented in this section, constitutes the second main result of this paper.



% ---------- SECTION -------------
% --------------------------------

\section{Numerical validation and implementation} \label{sec:num-val}

In this section we validate the analytic solutions through numerical evaluation and demonstrate their practical implementation using open-source software.
This serves two purposes, first, to verify that the solutions correctly satisfy the differential equations, and second, to show that the Weierstrass elliptic functions can be readily evaluated using standard numerical libraries.

We evaluate the analytic solutions using the \emph{pyweierstrass} Python package \cite{laurentPyweierstrass2022}, which provides Weierstrass elliptic functions as a wrapper around Jacobi elliptic functions from the \emph{mpmath} package \cite{MpmathPythonLibrary2023}. 
For comparison, we solve the differential equations numerically using the DOP853 Runge-Kutta algorithm from \emph{SciPy} \cite{virtanenSciPy10Fundamental2020}.

Figures \ref{fig:u-v-re} and \ref{fig:u-v-im} show the real and imaginary parts of the modal power $u_jv_j$ using the analytic solution \eqref{eq:uv-wp} compared against numerical integration of the abstract system \eqref{eq:uv-system}.
The analytic solutions (lines) are in excellent agreement with the numerical solutions (symbols).
The intermodal power conservation laws in \eqref{eq:power-conserved} are evident from the coordinated evolution of all four modes.

Figures \ref{fig:A-abs} and \ref{fig:A-phi} show the intensity and phase evolution of the physical field amplitudes $A_j$ in a representative four-wave mixing scenario.
The values were obtained by using \eqref{eq:u-v-quartic} to calculate $u_j, v_j$, then converting to $A_j, A_j^{*}$ via \eqref{eq:u-v-A}.
Again, the analytic solutions (lines) match the numerical integration (symbols) throughout the propagation distance.
The periodic energy exchange between modes visible in Figure \ref{fig:A-abs} is characteristic of phase-matched four-wave mixing, whilst Figure \ref{fig:A-phi} shows the phase evolution dominated by linear propagation with nonlinear contributions visible as regions of high curvature.

These comparisons confirm both the correctness of the analytic solutions and the feasibility of their numerical evaluation using readily available software tools.

\begin{figure}[H]
\centering
\includegraphics[width=1.0\textwidth]{plots/case1_plot_1.pdf}
\caption{Real part of abstract modal power $u_jv_j$: analytic solution \eqref{eq:uv-wp} (dashed lines) compared with numerical integration (symbols).}
\label{fig:u-v-re}
\end{figure}

\begin{figure}[H]
\centering
\includegraphics[width=1.0\textwidth]{plots/case1_plot_2.pdf}
\caption{Imaginary part of abstract modal power $u_jv_j$: analytic solution \eqref{eq:uv-wp} (dashed lines) compared with numerical integration (symbols).}
\label{fig:u-v-im}
\end{figure}

\begin{figure}[H]
\centering
\includegraphics[width=1.0\textwidth]{plots/case2_plot_1.pdf}
\caption{Intensity $|A_j|^2$ of physical field amplitudes: analytic solution \eqref{eq:u-v-quartic} converted via \eqref{eq:u-v-A} (dashed lines) compared with numerical integration (symbols).}
\label{fig:A-abs}
\end{figure}

\begin{figure}[H]
\centering
\includegraphics[width=1.0\textwidth]{plots/case2_plot_2.pdf}
\caption{Phase $\phi_j$ of physical field amplitudes: analytic solution \eqref{eq:u-v-quartic} converted via \eqref{eq:u-v-A} (dashed lines) compared with numerical integration (symbols).}
\label{fig:A-phi}
\end{figure}


\section{Conclusion}

We have presented complete analytic solutions for quasi-continuous-wave four-wave mixing in nonlinear optical fibres without any of the standard simplifying assumptions.
The solutions, expressed in terms of Weierstrass elliptic functions, provide the full complex field envelopes for all four waves under arbitrary initial conditions, including pump depletion, cross-phase modulation, and phase mismatch effects.
These solutions complement previous work in the literature by providing a unified treatment of the complex envelopes themselves rather than treating amplitudes and phases separately.

Guided by the structure of these solutions, we developed a sequence of coordinate transformations that reveal a canonical form of the four-wave mixing equations.
In these canonical coordinates, the system becomes parameter-free with a universal structure applying to all four-wave mixing processes regardless of specific fibre properties or operating conditions.
The solutions in canonical coordinates are single-valued meromorphic Kronecker theta functions, connecting four-wave mixing to a broader class of integrable nonlinear optical systems.
Remarkably, the transformations preserve the essential structure of the differential equations despite depending explicitly on the propagation variable, demonstrating a profound invariance property of four-wave mixing dynamics.

The connection between the Hamiltonian formulation and the Frobenius-Stickelberger determinant, established in Appendix~\ref{app:fs-det-ham-proof}, provides deeper insight into why these conservation laws hold and links four-wave mixing to classical elliptic function theory.
This mathematical framework may prove valuable for analysing related parametric processes and could extend to higher-order mixing or multimode interactions.

The methods developed here are broadly applicable to related nonlinear systems.
In particular, the approach may be transferable to the analysis of mode coupling in multimode nonlinear fibre optics (see, e.g., Chapter 4 in \cite{heskethNonlinearEffectsMultimode2014}), where similar coupled-mode structures arise.

The practical utility of these solutions has been demonstrated through numerical evaluation using open-source Python libraries, confirming both their validity and computational accessibility.
These analytic solutions may provide new insights for the design and optimisation of wavelength converters, parametric amplifiers, and quantum light sources by enabling exact analysis where approximations previously proved inadequate.


\section*{Acknowledgements}
The author thanks Maria Dimou for a careful read of the manuscript.

\appendix


% ---------- APPENDIX ------------
% --------------------------------

\section{Parameter definitions in the original coordinates}
\label{app:param-def-appendix}

This appendix provides explicit formulae for the parameters $b_l$, $c_l$, and $d_l$ that appear in the quartic polynomial \eqref{eq:drho-sqrd-3} and related expressions throughout Section~\ref{sec:mode-power}.
These parameters are expressed in terms of the system parameters $a_j$, $a_{j,k}$ and the conserved constants $\gamma_j$.

\begin{align}
{b}_{0} &= {a}_{0} + \sum_{j=1}^{4} {a}_{j} {\gamma}_{j} + \frac{1}{2}\sum\limits_{j,k=1}^{4} {a}_{j,k} {\gamma}_{j} {\gamma}_{k}, \notag \\
{b}_{1} &= - \sum_{j=1}^{4} {a}_{j} - \frac{1}{2}\sum\limits_{j,k=1}^{4} \left({\gamma}_{j} + {\gamma}_{k}\right) {a}_{j,k}, \notag \\
{b}_{2} &= \frac{1}{2}\sum\limits_{j,k=1}^{4} {a}_{j,k}, \notag \\
{c}_{0} &= \prod_{j=1}^{4} {\gamma}_{j}, \notag \\
{c}_{1} &= - \sum_{k=1}^{4} \prod_{\substack{j=1 \\ j \ne k}}^{4} {\gamma}_{j}, \notag \\
{c}_{2} &= \frac{3}{2} \sum_{j=1}^{4} {\gamma}_{j}^{2} - \frac{1}{4}\sum\limits_{j,k=1}^{4} \left({\gamma}_{j} - {\gamma}_{k}\right)^{2} = - \frac{1}{2}\sum_{j=1}^{4} {\gamma}_{j}^{2}, \notag \\
{d}_{0} &= {b}_{0}^{2} - 4 {c}_{0}, \notag \\
{d}_{1} &= 2 {b}_{0} {b}_{1} - 4 {c}_{1}, \notag \\
{d}_{2} &= 2 {b}_{0} {b}_{2} + {b}_{1}^{2} - 4 {c}_{2}, \notag \\
{d}_{3} &= 2 {b}_{1} {b}_{2}, \notag \\
{d}_{4} &= {b}_{2}^{2} - 4 
\end{align}

% ---------- APPENDIX ------------
% --------------------------------

\section{Initial condition relations}
\label{app:init-conds}

This appendix gives the relationships between initial conditions for the physical system and the constants $\gamma_j$ and $\rho(0)$ used throughout the analysis.
These relations enable determination of the constants from specified initial values of $u_j(0)$ and $v_j(0)$.

\begin{align}
\rho{\left(0 \right)} &= - \frac{1}{4}\sum_{j=1}^{4} u_j{\left(0 \right)} v_j{\left(0 \right)}, \notag \\
\left. \frac{d}{d z} \rho{\left(z \right)} \right|_{\substack{ z=0 }} &= \prod_{j=1}^{4} u_j(0) - \prod_{j=1}^{4} v_j(0), \notag \\
{\gamma}_{j} &= u_j{\left(0\right)} v_j{\left(0 \right)} + \rho{\left(0 \right)}
\end{align}


% ---------- APPENDIX ------------
% --------------------------------

\section{The Frobenius-Stickelberger determinant}
\label{app:fs-det-ham-proof}

The Frobenius-Stickelberger (FS) determinant formula is an elliptic function identity relating products of $\sigma$ functions to determinants of derivatives of $\wp$ \cite{whittakerCourseModernAnalysis2021, frobeniusZurTheorieElliptischen1877}.
In this appendix we demonstrate that the conservation of the Hamiltonian can be viewed as a manifestation of this classical identity, providing a deeper structural understanding of why the Hamiltonian is conserved in four-wave mixing.

Our starting point is equation \eqref{eq:fs-det-start}, which relates the wave mixing product to the phase modulation terms via the Hamiltonian.
We will show that this equation can be transformed into the FS determinant formula by treating the left-hand side (the product of modes) and right-hand side (the polynomial in $\rho$) separately, each ultimately expressed in terms of Weierstrass functions.

We are interested in the $4\times4$ case of FS in the form:
\begin{align}
\label{eq:fs-det-m}
\frac{C\,\sigma{\left(z + {\nu}_{1}\right)} \sigma{\left(z + {\nu}_{2}\right)} \sigma{\left(z + {\nu}_{3}\right)} \sigma{\left(z - {\nu}_{1} - {\nu}_{2} - {\nu}_{3}\right)}}{\sigma^{4}{\left(z\right)}} & = \det M
\end{align}
where:
\begin{align}
C = &\frac{12  \sigma{\left({\nu}_{1} - {\nu}_{2}\right)} \sigma{\left({\nu}_{1} - {\nu}_{3}\right)} \sigma{\left({\nu}_{2} - {\nu}_{3}\right)} }{\sigma{\left({\nu}_{1}\right)} \sigma^{4}{\left({\nu}_{2}\right)} \sigma^{4}{\left({\nu}_{3}\right)}} \\[6pt]
\det M = &
\begin{vmatrix}
1 & \wp(z) & \frac{\partial \wp(z)}{\partial z} & \frac{\partial^2 \wp(z)}{\partial z^2} \\
1 & \wp(\nu_1) & -\frac{\partial \wp(\nu_1)}{\partial \nu_1} & \frac{\partial^2 \wp(\nu_1)}{\partial \nu_1^2} \\
1 & \wp(\nu_2) & -\frac{\partial \wp(\nu_2)}{\partial \nu_2} & \frac{\partial^2 \wp(\nu_2)}{\partial \nu_2^2} \\
1 & \wp(\nu_3) & -\frac{\partial \wp(\nu_3)}{\partial \nu_3} & \frac{\partial^2 \wp(\nu_3)}{\partial \nu_3^2} \\
\end{vmatrix}  \notag \\[6pt]
= & - 6 \left(\wp{\left({\nu}_{1}\right)} - \wp{\left(z\right)}\right) \left(\wp{\left({\nu}_{2}\right)} - \wp{\left(z\right)}\right) \left(\wp{\left({\nu}_{1}\right)} - \wp{\left({\nu}_{2}\right)}\right) \wp'{\left({\nu}_{3}\right)} \notag \\
& + 6 \left(\wp{\left({\nu}_{1}\right)} - \wp{\left(z\right)}\right) \left(\wp{\left({\nu}_{3}\right)} - \wp{\left(z\right)}\right) \left(\wp{\left({\nu}_{1}\right)} - \wp{\left({\nu}_{3}\right)}\right) \wp'{\left({\nu}_{2}\right)}  \notag \\
& - 6 \left(\wp{\left({\nu}_{2}\right)} - \wp{\left(z\right)}\right) \left(\wp{\left({\nu}_{3}\right)} - \wp{\left(z\right)}\right) \left(\wp{\left({\nu}_{2}\right)} - \wp{\left({\nu}_{3}\right)}\right) \wp'{\left({\nu}_{1}\right)}  \notag \\
& - 6 \left(\wp{\left({\nu}_{1}\right)} - \wp{\left({\nu}_{2}\right)}\right) \left(\wp{\left({\nu}_{1}\right)} - \wp{\left({\nu}_{3}\right)}\right) \left(\wp{\left({\nu}_{2}\right)} - \wp{\left({\nu}_{3}\right)}\right) \wp'{\left(z\right)}
\end{align}

The starting point for us is the following relation established via \eqref{eq:ham-uv}, \eqref{eq:duv_j}, and \eqref{eq:Q-uv}, in which any $j$ applies in the RHS modal power derivative:
\begin{align}
2 \prod_{k=1}^{4} u_k &= {a}_{0} + \sum_{k=1}^{4} {a}_{k} u_k v_k  + \frac{1}{2}\sum\limits_{k,l=1}^{4} {a}_{k,l} u_k v_k u_l v_l - \frac{\partial}{\partial z} u_j\, v_j, \notag \\
&=\rho' + \sum_{l=0}^{2}b_l\,\rho\left(z\right)^l. \label{eq:fs-det-start}
\end{align}
We will show how \eqref{eq:fs-det-start} can be transformed into \eqref{eq:fs-det-m}

\subsection{The left hand side}
From \eqref{eq:uv-wp}, the right hand side of \eqref{eq:fs-det-start} is a doubly periodic elliptic function,
so too then is the left hand side. Thus for integers $m,n$ and half-periods $\omega_1, \omega_3$ we must have:
\begin{align}
\prod_{j=1}^{4} \frac{u_j{\left(z\right)}}{u_j{\left(2 m {\omega}_{3} + 2 n {\omega}_{1} + z\right)}} & = 1. \label{eq:u-prod-is-one}
\end{align}
We substitute our solution \eqref{eq:u-v-quartic} into \eqref{eq:u-prod-is-one} and use the quasi-periodicity of $\sigma$:
\begin{align}
\sigma{\left(2 m {\omega}_{3} + 2 n {\omega}_{1} + z\right)} = \left(-1\right)^{m n + m + n} \sigma{\left(z\right)} e^{\left(2 m {\omega}_{3} + 2 n {\omega}_{1} + 2 z\right) \zeta{\left(m {\omega}_{3} + n {\omega}_{1}\right)}} \label{eq:quasi-sigma}
\end{align}
to reduce the product in \eqref{eq:u-prod-is-one} to an exponential which, as it is equal to one, must have argument $2\pi i \mathrm{N}(n,m)$, with $\mathrm{N}(n,m)$ an integer that varies with $n,m$.
This leads to:
\begin{align}
2 i \pi \mathrm{N}{\left(n,m \right)} &= - \left(2 m {\omega}_{3} + 2 n {\omega}_{1}\right) \sum_{j=1}^{4} {r}_{0,j} - 2 \zeta{\left(m {\omega}_{3} + n {\omega}_{1}\right)} \sum_{j=1}^{4} {\nu}_{j}, \label{eq:exp-arg-is-one} \\ 
\sum_{j=1}^{4} {\nu}_{j} &= 2 \mathrm{N}{\left(1,0 \right)} {\omega}_{3} - 2 \mathrm{N}{\left(0,1 \right)} {\omega}_{1} = 0 \pmod{\text{lattice}}, \label{eq:nuj-sum-zero}  \\ 
\sum_{j=1}^{4} {r}_{0,j} &= 2 \,\zeta{\left(\mathrm{N}{\left(0,1 \right)} {\omega}_{1} - \mathrm{N}{\left(1,0 \right)} {\omega}_{3}\right)} \label{eq:roj-is-zeta} 
\end{align}
where $\nu_j=\mu_j-z_0$, and in moving from \eqref{eq:exp-arg-is-one} to \eqref{eq:nuj-sum-zero} and \eqref{eq:roj-is-zeta} we made use of the identity $\zeta{\left({\omega}_{3}\right)} {\omega}_{1} - \zeta{\left({\omega}_{1}\right)} {\omega}_{3} = i \pi /2$. 
Ultimately, equations \eqref{eq:quasi-sigma}, \eqref{eq:nuj-sum-zero} and \eqref{eq:roj-is-zeta} allow us to make the following substitution in the left hand side of \eqref{eq:fs-det-start}:
\begin{align}
\frac{\sigma{\left(z + {\nu}_{4} \right)}}{\sigma{\left({\nu}_{4} \right)}}\,\exp{\left(z \sum\limits_{j=1}^{4} {r}_{0,j}\right)} = - \frac{\sigma{\left(z - {\nu}_{1} - {\nu}_{2} - {\nu}_{3} \right)}}{\sigma{\left({\nu}_{1} + {\nu}_{2} + {\nu}_{3} \right)}}.
\end{align}

\subsection{The right hand side}
The following algebraic identity follows from Waring-Lagrange interpolation and holds for any values of the variables and functions:
\begin{align}
\sum_{l=0}^{2}b_l\,\rho\left(z\right)^l =\, &\frac{\left({\gamma}_{1} - \rho{\left(z \right)}\right) \left({\gamma}_{2} - \rho{\left(z \right)}\right)}{\left({\gamma}_{1} - {\gamma}_{3}\right) \left({\gamma}_{2} - {\gamma}_{3}\right)} \sum\limits_{l=0}^{2}b_l\,\gamma_1^l + \notag \\
&\frac{\left({\gamma}_{1} - \rho{\left(z \right)}\right) \left({\gamma}_{3} - \rho{\left(z \right)}\right)}{\left({\gamma}_{1} - {\gamma}_{2}\right) \left({\gamma}_{3} - {\gamma}_{2}\right)} \sum\limits_{l=0}^{2}b_l\,\gamma_2^l + \notag \\
&\frac{\left({\gamma}_{2} - \rho{\left(z \right)}\right) \left({\gamma}_{3} - \rho{\left(z \right)}\right)}{\left({\gamma}_{1} - {\gamma}_{2}\right) \left({\gamma}_{1} - {\gamma}_{3}\right)} \sum\limits_{l=0}^{2}b_l\,\gamma_3^l \label{eq:waring-lagrange}
\end{align}
where in our case $\gamma_j=\rho(\mu_j)$ and the sum over $b_l \gamma_j$ can be written using \eqref{eq:drho-sqrd-2} as:
\begin{align}
\sum\limits_{l=0}^{2}b_l\,\gamma_j^l = \rho'\left(\mu_j\right).
\end{align}
We can then use \eqref{eq:uv-wp} to express $\rho$ in terms of $\wp$ so that we can substitute all variables in \eqref{eq:waring-lagrange} for values in terms of $\wp, \wp'$. 
We then divide both sides of \eqref{eq:fs-det-start} by:
\begin{align}
\frac{\wp'{\left(z_{1}\right)}}{\sqrt{{d}_{4}} \left(\wp{\left(z_{1}\right)} - \wp{\left(z - z_{0}\right)}\right)^2 }
\end{align}
and observe that it takes the equivalent form to \eqref{eq:fs-det-m} up to constant factors that may look different but can be shown to be identical by multiplying by $\sigma(z)^4$ and taking $z\rightarrow 0$.

This completes the demonstration that the Hamiltonian conservation law \eqref{eq:fs-det-start} is equivalent to the Frobenius-Stickelberger determinant formula.
This connection reveals that the conserved Hamiltonian of four-wave mixing is not merely a convenient construct but rather a manifestation of deep structural properties of elliptic functions.
The FS determinant framework may provide tools for generalizing these results to higher-order wave mixing processes or systems with additional degrees of freedom.

% ---------- APPENDIX ------------
% --------------------------------

\section{Parameter definitions in the bar coordinates}
\label{app:bar-coords-params}

This appendix provides explicit formulae for the parameters that appear in the bar coordinate system introduced in Section~\ref{subsec:bar-coords}, including the key transformation parameters $d_5$, $\varsigma$, $\chi$, and the rescaled constants $\bar{\gamma}_j$, $\theta_j$, and $\bar{\kappa}_j$.

\begin{align}
{d}_{5} &= {d}_{1} + 2 {d}_{2} {\lambda}_{1} + 3 {d}_{3} {\lambda}_{1}^{2} + 4 {d}_{4} {\lambda}_{1}^{3}, \notag  \\
\varsigma & = \frac{1}{2} \left(b_0 + b_1 {\lambda}_{1} + b_2{\lambda}_{1}^2\right) \left(\sqrt{\prod\limits_{j=1}^4 \gamma_j - \lambda_1}\right)^{-1}  = \pm 1, \\
\chi &= \frac{8\left(b_0 + b_1 {\lambda}_{1} + b_2{\lambda}_{1}^2\right)}{d_5} = \frac{16\, \varsigma}{d_5}\sqrt{\prod\limits_{j=1}^4 \gamma_j - \lambda_1}, \notag \\
\bar{\gamma}_{j} &= \frac{{d}_{5}}{4 \left({\gamma}_{j} - {\lambda}_{1}\right)} + \frac{{b}_{1} + 2 {b}_{2} {\lambda}_{1}}{\chi} - \frac{4}{\chi^{2}}, \notag \\
{\theta}_{j} &= \frac{b_0 + b_1 {\lambda}_{1} + b_2{\lambda}_{1}^2}{2\left({\gamma}_{j} - {\lambda}_{1}\right)} + \frac{{b}_{1}}{4} + \frac{{b}_{2} {\gamma}_{j}}{2} + \frac{{b}_{2} {\lambda}_{1}}{2} - \frac{1}{\chi}, \quad \sum\limits_{j=1}^{4} \theta_j = 0, \notag \\
\bar{\kappa}_j &= 2 {\theta}_{j} + \frac{1}{\chi} - {b}_{2} {\gamma}_{j} - \zeta{\left({\mu}_{j} - z_{0}, g_2, g_3\right)}.
\end{align}

% ---------- APPENDIX ------------
% --------------------------------

\section{Scaled parameters in the tilde coordinates}
\label{app:scaled-params}
When the system in subsection \ref{subsec:tilde-coords} (tilde notation) is considered a transform of that in subsection \ref{subsec:bar-coords} (bar notation), then parameters are related via the following scaling laws:
\begin{align}
\tilde{H} &= \chi^3\bar{H}, \\
\tilde{\epsilon}_j &= \epsilon_j \sqrt[4]{\varsigma}, \notag \\
\tilde{\gamma}_j &= \chi^2\bar{\gamma}_j, \notag \\
\tilde{g}_2 &= \chi^4 g_2, \notag \\
\tilde{g}_3 &= \chi^6 g_3, \notag \\
\xi_0 &= \frac{z_0}{\chi}, \notag \\
\tilde{\mu}_j &= \frac{\mu_j}{\chi}.
\end{align}

If the system in subsection \ref{subsec:tilde-coords} is considered as a standalone system irrespective of any transformation,
then the paramters can be given exclusively in the coordinate system denoted by tilde notation as follows.
Firstly, $\tilde{H}$ can be obtained from substituting $\tilde{u}_j{\left(0\right)}, \tilde{v}_j{\left(0 \right)}$ into \eqref{eq:H-tilde}
and $\tilde{\epsilon}_j$ can be considered an integration constant fixed by initial conditions to capture any phase offset between $\tilde{u}_j, \tilde{v}_j$. 
We may further use relations:
\begin{align}
\tilde{\gamma}_{j} =& \tilde{u}_j{\left(0\right)} \tilde{v}_j{\left(0 \right)} - \frac{1}{4}\sum_{j=1}^{4} \tilde{u}_j{\left(0\right)} \tilde{v}_j{\left(0\right)}, \notag \\
\tilde{g}_{2} =& \frac{\tilde{H}^{2}}{12} + \tilde{H} \left(\frac{{\tilde{e}}_{2}}{12} + \frac{16}{3}\right) + \frac{{\tilde{e}}_{2}^{2}}{48} + \frac{2 {\tilde{e}}_{2}}{3} - \frac{{\tilde{e}}_{3}}{4} + \frac{64}{3}, \notag \\
\tilde{g}_{3} =& \frac{\tilde{H}^{3}}{216} + \tilde{H}^{2} \left(\frac{{\tilde{e}}_{2}}{144} - \frac{5}{9}\right) + \tilde{H} \left(\frac{{\tilde{e}}_{2}^{2}}{288} - \frac{2 {\tilde{e}}_{2}}{9} - \frac{{\tilde{e}}_{3}}{48} - \frac{64}{9}\right) \notag \\
&+ \frac{{\tilde{e}}_{2}^{3}}{1728} - \frac{5 {\tilde{e}}_{2}^{2}}{144} - \frac{{\tilde{e}}_{2} {\tilde{e}}_{3}}{96} - \frac{8 {\tilde{e}}_{2}}{9} + \frac{{\tilde{e}}_{3}}{3} + \frac{{\tilde{e}}_{4}}{4} - \frac{512}{27}
\end{align}
where $\tilde{e}_j=\tilde{e}_j\left(\tilde{\gamma}_1, \tilde{\gamma}_2, \tilde{\gamma}_3, \tilde{\gamma}_4\right)$ is the $j^{th}$ order \emph{elementary symmetric polynomial} formed with the four $\tilde{\gamma}_j$. The points $\xi_0, \tilde{\mu}_j$ are found by inverting:
\begin{align}
\wp{\left(\xi_{0},\tilde{g}_{2},\tilde{g}_{3} \right)} &= - \frac{\tilde{H}}{12} - \frac{{\tilde{e}}_{2}}{24} - \frac{\sum_{j=1}^{4} \tilde{u}{\left(0,{\mu}_{j} \right)} \tilde{v}{\left(0,{\mu}_{j} \right)}}{4} + \frac{4}{3}, \notag \\
\wp'{\left(\xi_{0},\tilde{g}_{2},\tilde{g}_{3} \right)} &= \frac{1}{4}\left(\prod_{j=1}^{4}{\tilde{u}_j(0)} - \prod_{j=1}^{4}{\tilde{v}_j(0)}\right), \notag \\
\wp{\left({\tilde{\mu}}_{j} - \xi_{0},\tilde{g}_{2},\tilde{g}_{3} \right)} &= - \frac{\tilde{H}}{12} - \frac{{\tilde{e}}_{2}}{24} + {\tilde{\gamma}}_{j} + \frac{4}{3}, \notag \\
\wp'{\left({\tilde{\mu}}_{j} - \xi_{0},\tilde{g}_{2},\tilde{g}_{3} \right)} &= \tilde{H} + \frac{{\tilde{e}}_{2}}{4} - \frac{{\tilde{\gamma}}_{j}^{2}}{2} - 4 {\tilde{\gamma}}_{j}.
\end{align}

\printbibliography

\end{document}
